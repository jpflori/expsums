\documentclass[11pt,a4paper]{article}

\usepackage[english]{babel}
\usepackage[utf8]{inputenc}
\usepackage{lmodern}
\usepackage[T1]{fontenc}
\usepackage{datetime}
\usepackage{xifthen}

\usepackage{amsmath,amssymb}
\usepackage{mathtools}
\usepackage{geometry}
\usepackage{hyperref}
\usepackage{microtype}
\usepackage{verbatim}
\usepackage{rotating}
\usepackage{array}
\usepackage[ruled,linesnumbered]{algorithm2e}
\usepackage{tikz}
\usepackage{pgfplots}

\usepackage{xspace}
\newcommand{\ie}{i.e.\@\xspace}
\newcommand{\eg}{e.g.\@\xspace}
\newcommand{\etal}{et al.\@\xspace}

\usepackage{amsthm}
\newtheorem{theorem}{Theorem}[section]
\newtheorem{definition}[theorem]{Definition}
\newtheorem{proposition}[theorem]{Proposition}
\newtheorem{remark}[theorem]{Remark}
\newtheorem{example}[theorem]{Example}
\newtheorem{corollary}[theorem]{Corollary}
\newtheorem{lemma}[theorem]{Lemma}
\newtheorem{conjecture}[theorem]{Conjecture}

\newcommand{\N}{\mathbb{N}}
\newcommand{\Z}{\mathbb{Z}}
\newcommand{\Q}{\mathbb{Q}}
\newcommand{\GF}[2][2]{\mathbb{F}_{#1^{#2}}}
\newcommand{\T}{\mathcal{T}}
\newcommand{\set}[1]{\left\{ #1 \right\}}
\newcommand{\card}[1]{\left| #1 \right|}
\DeclareMathOperator{\lcm}{lcm}
\DeclareMathOperator{\Tr}{Tr}
\makeatletter
\newcommand{\tr}[3][1]{\ifthenelse{\isempty{#3}}%
  {\Tr_{#1}^{#2}}%
  {\Tr_{#1}^{#2}\left(#3\right)}}
\newcommand{\addch}[1]{\ifthenelse{\isempty{#1}}%
  {\chi}%
  {\chi \left( #1 \right)}}
\newcommand{\mulch}[2][m_1]{\ifthenelse{\isempty{#2}}%
  {\psi_{#1}}%
  {\psi_{#1} \left( #2 \right)}}
\makeatother
\newcommand{\Wa}[1]{\widehat{\chi_{#1}}}
\newcommand{\Snu}[1][\nu]{S_{#1}(a, b, \omega)}
\newcommand{\mystery}{h(a, b, w_1, w_2)}

\hypersetup{pdftitle={A conjecture about Gauss sums and bentness of binomial Boolean functions},%
  pdfauthor={Jean-Pierre Flori},%
  pdfsubject={Mathematics and Cryptography},%
  pdfcreator={Jean-Pierre Flori},%
  pdfproducer={Jean-Pierre Flori},%
  pdfkeywords={mathematics} {cryptography} {Boolean functions} {bent functions} {Walsh spectrum} {Kloosterman sums},
}

\title{A conjecture about Gauss sums and bentness of binomial Boolean functions}

\author{Jean-Pierre Flori
  \thanks{ANSSI (Agence nationale de la sécurité des systèmes d'information),
    51, boulevard de La Tour-Maubourg,
    75700 Paris 07 SP, France.
    \texttt{jean-pierre.flori@ssi.gouv.fr}}
}

%\date{\today~--~\currenttime}
\date{}

\begin{document}

\maketitle

\begin{abstract}
  In this note, the polar decomposition of binary fields of even extension degree is
  used to reduce the evaluation of the Walsh transform of binomial Boolean functions
  to that of Gauss sums.
  In the case of extensions of degree four times an odd number, an explicit formula involving a Kloosterman sum is conjectured,
  proved with further restrictions, and supported by extensive experimental
  data in the general case.
  In particular, the validity of this formula is shown to be equivalent
  to a simple and efficient characterization for bentness
  previously conjectured by Mesnager.
\end{abstract}

\noindent
{\bf Keywords}. Boolean functions, bent functions, Walsh spectrum, exponential sums, Gauss sums, Kloosterman sums.


\section{Introduction}
\label{sec:introduction}

Bent functions are Boolean functions defined over an extension of even
degree and achieving optimal non-linearity.
They are of both combinatorial and cryptographic interest.
Unfortunately, characterizing bentness of an arbitrary Boolean function
is a difficult problem,
and even the less general question of providing simple and efficient
criteria within infinite families of functions in a specific polynomial form
is still challenging.

For a Boolean function $f$ defined over $\GF{n}$ with $n = 2 m$ and
given in polynomial form, a classical characterization for bentness
is that its Walsh transform $\Wa{f}$ values are only $2^{\pm m}$.
Nevertheless, such a characterization is neither concise nor efficient:
the best algorithm to compute the full Walsh spectrum has complexity $O(n 2^n)$,
which is asymptotically optimal.
Whence the need to restrict to functions
in a given form and to look for more efficient criteria.
Unfortunately, only a few infinite families of Boolean functions
with a simple and efficient criterion for bentness are known.

The most classical family is due to Dillon~\cite{MR2624542}
and is made of monomial functions:
\[
f_a(x) = \tr{n}{a x^{r(2^m-1)}} \enspace ,
\]
where $n = 2 m$, $a \in \GF{n}^*$ and $r$ is co-prime with $2^m + 1$.
Such functions are bent (and even hyper-bent) if and only if the Kloosterman sum $K_m(a)$
associated with $a$ is equal to zero~\cite{MR2624542,DBLP:journals/tit/Leander06,DBLP:journals/tit/CharpinG08}.
Not only does such a criterion gives a concise and elegant characterization for bentness,
but using the connection between Kloosterman sums and elliptic curves~\cite{MR925289,MR1054286}
it also allows to check for bentness in polynomial time~\cite{DBLP:conf/seta/Lisonek08,DBLP:journals/corr/abs-1104-3882}.
Further results on Kloosterman sums involving $p$-adic arithmetic~\cite{MR2794931,6126036,Moloney:PHD}
lead to even faster generation of zeros of Kloosterman sums and so of (hyper-)bent functions.

Mesnager~\cite{DBLP:journals/dcc/Mesnager11} proved a similar criterion
for a family Boolean functions in binomial form:
\[
f_{a,b}(x) = \tr{n}{a x^{r(2^m-1)}} + \tr{2}{b x^{\frac{2^n-1}{3}}} \enspace ,
\]
where $n = 2 m$, $a \in \GF{n}^*$, $b \in \GF[4]{}^*$
and $r$ is co-prime with $2^m + 1$
(but also $r = 3$ which divides $2^m+1$~\cite{DBLP:conf/ima/Mesnager09}).
When the extension degree $n$ is twice an odd number, that is when $m$ is odd,
$f_{a,b}$ is (hyper-)bent if and only if $K_m(a) = 4$.
Unfortunately, the proof does not extend to the case where $m$ is even.
Nevertheless, it is easy to show that $K_m(a) = 4$ is still a necessary
condition for $f_{a,b}$ to be bent in this latter case
(but note that $f_{a,b}$ can no longer be hyper-bent).
Further experimental evidence gathered by Flori, Mesnager
and Cohen~\cite{DBLP:journals/iacr/FloriMC11} supported the conjecture
that it should also be a sufficient condition:
for $m$ up to $16$, $f_{a,b}$ is bent if and only if $K_m(a) = 4$.

In this note, the polar decomposition of fields of even extension degree
$n = 2^\nu m$ with $m$ odd is used to reduce the evaluation of the Walsh transform
of $f_{a,b}$ at $\omega \in \GF{n}^*$ to that of a Gauss sum of the form
\begin{align}
\label{eqn:gengauss}
\sum_{u \in U} \mulch[n]{b \tr[m]{n}{\omega u}} \addch{\tr{n}{a u^{2^{2^{\nu-1}m}-1}}} \enspace ,
\end{align}
where $\GF{n}^*$ is decomposed as $\GF{n}^* \simeq U \times \GF{m}^*$,
$\mulch[n]{}$ is a cubic multiplicative character
and $\addch{}$ a quadratic additive character.
In the case of extensions of degree four times an odd number,
that is when $m$ is twice an odd number,
an explicit formula involving the Kloosterman sum $K_m(a)$ is proved
for $\omega$ lying in the subfield $\GF{2m}$,
and conjectured and supported by extensive experimental evidence
when $\omega \in \GF{4m}$,
In particular, the validity of this formula would prove the characterization
conjectured by Mesnager for extensions of degree four times an odd number
and give hope to prove the conjecture for $n$ of any $2$-adic valuation.
\begin{conjecture}%[\cite{DBLP:journals/dcc/Mesnager11}]
\label{cnj:kloofour}
Let $n = 4m$ with $m$ odd, $a \in \GF{2m}^*$ and $b \in \GF[4]{}^*$.
The function $f_{a,b}$ is bent if and only if $K_{m_1}(a) = 4$.
\end{conjecture}

\section{Notation}
\label{sec:notation}

\subsection{Field trace}

For extension degrees $m$ and $n$ such that $m$ divides $n$,
the field trace from $\GF{n}$ down to $\GF{m}$ is denoted by $\tr[m]{n}{x}$.

\subsection{Polar decomposition}

Let $n \geq 2$ be an even integer and
$\nu \geq 1$ denote its $2$-adic valuation.
We denote by $m_i$ for $0 \leq i \leq \nu$ the integer $n / 2^i$,
\eg $m_0 = n$ and $m_\nu = m$ with the notation of the previous section.

For $0 \leq i < \nu$, the multiplicative group  $\GF{m_i}^*$
can be split using the so-called polar decomposition
\begin{align*}
\GF{m_i}^* & \simeq U_{i+1} \times \GF{m_{i+1}}^* \enspace ,
\end{align*}
where $U_{i+1} \subset \GF{m_i}^*$ is the subgroup of $(2^{m_{i+1}}+1)$-th roots of unity
and $\GF{m_{i+1}}^*$ the subgroup of $(2^{m_{i+1}}-1)$-th roots of unity.
Repeating this construction, $\GF{m_0}^*$ can be decomposed as
\begin{align*}
\GF{m_0}^* & \simeq U_1 \times \cdots \times U_\nu \times \GF{m_\nu}^* \enspace .
\end{align*}
Let $U$ denote the image of $U_1 \times \cdots \times U_\nu$ within $\GF{m_0}^*$,
that is the set of $(2^{m_1}+1) \cdots (2^{m_\nu}+1)$-th roots of unity.
Then $\GF{m_0}^*$ can be decomposed as
\begin{align*}
\GF{m_0}^* & \simeq U \times \GF{m_\nu}^* \enspace .
\end{align*}

Remark that $3$ divides $2^{m_\nu}+1$ and is coprime with $2^{m_\nu}-1$ and $2^{m_i}+1$ for $0 \leq i < \nu$.
Therefore the function $x \mapsto x^3$ is a permutation of $\GF{m_\nu}^*$ and
$U_i$ for $1 \leq i < \nu$, and $3$-to-$1$ on $U_\nu$.

\subsection{Hilbert's Theorem 90}

For $1 \leq i \leq \nu$ and $j \in \GF{}$, let $\T_{m_i}^j$ be the set
\begin{align*}
\T_{m_i}^j & = \set{x \in \GF{m_i}, \tr{m_i}{x^{-1}} = j}
\end{align*}
of elements of $\GF{m_i}$ whose inverses have trace $j$
(defining $0^{-1}$ to be $0$).

Hilbert's Theorem 90 implies that the function $x \mapsto x + x^{-1}$ is
$2$-to-$1$ from $U_i \backslash \set{1}$ to $\T_{m_i}^1$
and from $\GF{m_i}^* \backslash \set{1}$ to $\T_{m_i}^0 \backslash \set{0}$
(and both $0$ and $1$ are sent onto $0$).

\subsection{Dickson polynomials}
\label{sec:dickson}

Let $D_3(x) = x^3 + x$ be the third Dickson polynomial of the first kind.
A notable property of $D_3$ is that $D_3(x + x^{-1}) = x^3 + x^{-3}$.
It implies in particular that $D_3$ induces a permutation of $\T_{m_1}^0$
when $m_1$ is odd and of $\T_{m_1}^1$ when $m_1$ is even.

\subsection{Characters}

The non-principal quadratic additive character $\addch{}$ of $\GF{}$
together with the field trace can be used to construct all quadratic additive
characters of $\GF{m_i}$ for any $0 \leq i \leq \nu$.

The non-principal cubic multiplicative character $\mulch[m_i]{}$ of $\GF{m_i}$
for any $0 \leq i < \nu$ is defined for $x \in \GF{m_i}$ as
\[
\mulch[m_i]{x} = x^{\frac{2^{m_i}-1}{3}} \enspace .
\]
Note that if $x$ lies in a subextension,
that is $x \in \GF{m_{i+j}}$ with $0 \leq i+j < \nu$, then
\[
\mulch[m_i]{x} = \mulch[m_{i+j}]{x}^{2^j} \enspace .
\]
Note that for $i = \nu$, every $x \in \GF{m_\nu}$ is a cube.

\subsection{Walsh transform}

The Walsh transform of a Boolean function $f$ at $\omega \in \GF{m_0}$ is
\begin{align*}
\Wa{f}(\omega) & = \sum_{x \in \GF{m_0}} \addch{f(x) + \tr{m_0}{\omega x}} \enspace .
\end{align*}
The function $f$ is bent if and only if $\Wa{f}(\omega) = 2^{\pm m_1}$
for all $\omega \in \GF{m_0}$.

\subsection{Kloosterman sums}
\label{sec:kloo}

For $a \in \GF{m_1}$, the Kloosterman sum $K_{m_1}(a)$ is
\begin{align*}
K_{m_1}(a) & = \sum_{x \in \GF{m_1}} \addch{\tr{m_1}{a x + x^{-1}}} \enspace .
\end{align*}
It is easy to show that
\begin{align*}
\sum_{u_1 \in U_1} \addch{\tr{m_0}{a u_1}}
& = 1 + 2 \sum_{t \in \T_{m_1}^1} \addch{\tr{m_1}{a t}} \\
& = 1 - 2 \sum_{t \in \T_{m_1}^0} \addch{\tr{m_1}{a t}} \\
& = 1 - K_{m_1}(a) \enspace .
\end{align*}

\subsection{Cubic sums}
\label{sec:cubic}

For $a, b \in \GF{m_1}$, the cubic sum $C_{m_1}(a, b)$ is
\begin{align*}
C_{m_1}(a, b) & = \sum_{x \in \GF{m_1}} \addch{\tr{m_1}{a x^3 + b x}}
\end{align*}
The possible values of $C_{m_1}(a, b)$ were determined by Carlitz~\cite{MR544577}
together with simple criteria involving $a$ and $b$.

The most important consequence of Carlitz's results in our context
is that $C_{m_1}(a, a) = \sum_{x \in \GF{m_1}} \addch{\tr{m_1}{a D_3(x)}} = 0$ if and only if
\begin{itemize}
\item $\tr{m_1}{\alpha} = 0$ for $\alpha \in \GF{m_1}^*$ such that $a = \alpha^3$
when $m_1$ is odd (in that case $a$ is always a cube);
\item and when there exists $\alpha \in \GF{m_1}^*$ such that $a = \alpha^3$
(that is $a$ is a cube or equivalently $\mulch{a} = 1$)
and $\tr[2]{m_1}{\alpha} \neq 0$ (that is the cube root's half-trace is non zero)
when $m_1$ is even.
\end{itemize}
Charpin \etal later deduced that both in the odd case~\cite{DBLP:journals/jct/CharpinHZ07}
and in the even case~\cite{4595463,DBLP:journals/dm/CharpinHZ09}
these conditions are equivalent to $K_{m_1}(a) \equiv 1 \pmod{3}$.

For completeness, the other possible values for $C_{m_1}(a, a)$ when $m_1$ is even follow:
\begin{itemize}
\item When $a$ is a cube and $\tr[2]{m_1}{\alpha} = 0$,
then $C_{m_1}(a, a) = 2^{m_2 + 1} \addch{\tr{m_1}{u_0^3}}$,
where $u_0$ is any solution to $u^4 + u = \alpha^4$,
that is $u_0 = \sum_{i=0}^{(m_2-3)/2} \alpha^{4^{2*i+2}} + \gamma$
for any $\gamma \in \GF[4]{}$.
\item When $a$ is not a cube,
then $C_{m_1}(a, a) = - 2^{m_2} \addch{\tr{m_1}{a u_0^3}}$,
where $u_0$ is the unique solution to $u^4 + u / a = 1$,
that is $u_0 = \mulch{a} \sum_{i = 0}^{m_2-1} a^{4^i} a^{(4^i - 1)/3}$.
%$u_0 = a^{(4^{m_2}-1)/3} \sum_{i = 0}^{m_2-1} a^{4^i} a^{(4^i - 1)/3}$.
\end{itemize}

Finally, the following result on $C_{m_1}(a, 0)$ when $m_1 = 2 m_2$ is even will also be used:
\begin{align*}
C_{m_1}(a, 0)
& = \left\{
\begin{array}{ll}
(-1)^{m_2+1} 2^{m_2+1} & \text{if $\mulch{a} = 1$;} \\
(-1)^{m_2} 2^{m_2} & \text{if $\mulch{a} \neq 1$.} \\
\end{array}
\right.
\end{align*}

\subsection{Binomial functions}

The binomial Boolean functions $f_{a,b}$ studied in this note are defined over $\GF{m_0}$
and given in binomial form:
\begin{align*}
f_{a,b}(x)
%& = \tr{m_0}{a x^{2^{m_1}-1}} + \tr{2}{b x^{\frac{2^{m_0}-1}{3}}} \\
& = \tr{m_0}{a x^{2^{m_1}-1}} + \tr{2}{b \mulch[m_0]{x}} \enspace ,
\end{align*}
where $a \in \GF{m_0}$ and $b \in \GF[4]{}^*$.

We also define $f_a = f_{a,0}$ (corresponding to Dillon's monomial) and
$g_b(x) = \tr{2}{b \mulch[m_0]{x}}$.

\section{Preliminaries}

\subsection{Field of definition of the coefficients}

First notice that it is enough to know how to evaluate the Walsh transform of
functions $f_{a,b}$ for $a \in \GF{m_1}^*$ to extend results to $a \in \GF{m_0}^*$.
\begin{lemma}
Let $a \in \GF{m_0}^*$ be written as $a = \alpha \tilde{a}$
with $\alpha \in U_1$ and $\tilde{a} \in \GF{m_1}^*$
using the polar decomposition of $\GF{m_0}^*$.
Let $\tilde{\alpha} \in U_1$ be a squareroot of $\alpha$
and $\beta \in \GF[4]{}^*$ be $\beta = \mulch[m_0]{\alpha}^{-1}$.
Then
\begin{align*}
\Wa{f_{a,b}}(\omega) & = \Wa{f_{\tilde{a},\beta b}}(\tilde{\alpha} \omega) \enspace .
\end{align*}
\end{lemma}
\begin{proof}
Indeed, $x \mapsto \tilde{\alpha} x$ induces a permutation of $\GF{m_0}$,
and $\tilde{\alpha}^{2^{m_1}-1} = \tilde{\alpha}^{-2} = \alpha^{-1}$,
so that
\begin{align*}
\Wa{f_{a,b}}(\omega) & = \sum_{x \in \GF{m_0}} \addch{f_{a,b}(x) + \tr{m_0}{\omega x}} \\
& = \sum_{x \in \GF{m_0}} \addch{f_{a,b}(\tilde{\alpha} x) + \tr{m_0}{\omega \tilde{\alpha} x}} \\
& = \sum_{x \in \GF{m_0}} \addch{f_{\tilde{a},\beta b}(x) + \tr{m_0}{\omega \tilde{\alpha} x}} \\
& = \Wa{f_{\tilde{a},\beta b}}(\tilde{\alpha} \omega) \enspace . \qedhere
\end{align*}
\end{proof}
From now on we can suppose that $a \in \GF{m_1}^*$ without loss of generality.

\subsection{Polar decomposition}

We now split the sum expressing the Walsh transform of $f_{a,b}$ at $\omega \in \GF{m_0}$
using the polar decomposition of $\GF{m_0}^*$ as $\GF{m_0}^* = U_1 \times \cdots \times U_\nu \times \GF{m_\nu}^* = U \times \GF{m_\nu}^*$.
We write $x \in \GF{m_0}^*$ as $x = u_1 \cdots u_\nu y = u y$ for $u_i \in U_i$,
$u = u_1 \cdots u_\nu \in U$, and $y \in \GF{m_\nu}^*$.
The Walsh transform of $f_{a,b}$ at $\omega \in \GF{m_0}$ can then be written
\begin{align*}
\Wa{f_{a,b}}(\omega) & = \sum_{x \in \GF{m_0}} \addch{f_{a,b}(x) + \tr{m_0}{\omega x}} \\
& = 1 + \sum_{x \in \GF{m_0}^*} \addch{f_{a,b}(x) + \tr{m_0}{\omega x}} \\
& = 1 + \sum_{(u, y) \in U \times \GF{m_\nu}^*} \addch{f_{a,b}(u y)} \addch{\tr{m_0}{\omega u y}} \enspace .
\end{align*}
Note that $3$ divides $2^{m_\nu}+1$ so that $\frac{2^{m_0}-1}{3} = (2^{m_\nu}-1) \frac{2^{m_\nu}+1}{3} \prod_{i=1}^{\nu-1}(2^{m_i}+1)$ and $f_{a,b}(u y) = f_{a,b}(u)$.
Therefore
\begin{align*}
\Wa{f_{a,b}}(\omega) & = 1 + \sum_{u \in U} \addch{f_{a,b}(u)} \sum_{y \in \GF{m_\nu}^*} \addch{\tr{m_\nu}{\tr[m_\nu]{m_0}{\omega u} y}} \enspace .
\end{align*}
The sum ranging over $\GF{m_\nu}^*$ is equal to $-1$ when $\tr[m_\nu]{m_0}{\omega u} \neq 0$ and $2^{m_1}-1$ when $\tr[m_\nu]{m_0}{\omega u} = 0$.

When $\omega = 0$, the Walsh transform is
\begin{align}
\Wa{f_{a,b}}(0) & = 1 + \left( 2^{m_\nu} - 1 \right) \sum_{u \in U} \addch{f_{a,b}(u)} \enspace . \label{eqn:walshzero}
\end{align}

When $\omega \in \GF{m_0}^*$, write its polar decomposition $\omega = w_1 \cdots w_\nu o = w o$
where $(w_1, \ldots, w_\nu, o) \in U_1 \times \cdots U_\nu \times \GF{m_\nu}^*$
and $w = w_1 \cdots w_\nu \in U$.
The Walsh transform is then
\begin{align}
\Wa{f_{a,b}}(\omega) & = 1 - \sum_{u \in U, \tr[m_\nu]{m_0}{\omega u} \neq 0} \addch{f_{a,b}(u)} + (2^{m_\nu} - 1) \sum_{u \in U, \tr[m_\nu]{m_0}{\omega u} = 0} \addch{f_{a,b}(u)} \nonumber \\
& = 1 - \sum_{u \in U} \addch{f_{a,b}(u)} + 2^{m_\nu} \sum_{u \in U, \tr[m_\nu]{m_0}{\omega u} = 0} \addch{f_{a,b}(u)} \enspace . \label{eqn:walshunit}
\end{align}
To go further, the cases $\nu = 1$ and $\nu \geq 2$ have to be dealt with separately.

\section{Odd case}
\label{sec:odd}

In this section, it is supposed that $\nu = 1$, \ie $m_1$ is odd and $U = U_1$,
which is the case that Mesnager settled~\cite{DBLP:journals/dcc/Mesnager11}.
We recall the main ingredients and results of her work as
similar ideas will be used for the even case.

For $\omega \neq 0$, $\tr[m_1]{m_0}{\omega u_1} = 0$ if and only if $u_1 = w_1^{-1}$, so that
\begin{align*}
\sum_{u_1 \in U_1, \tr[m_1]{m_0}{\omega u_1} = 0} \addch{f_{a,b}(u_1)}
& = \addch{f_{a,b}(w_1^{-1})} \enspace .
\end{align*}

The only difficulty lies in the computation of $\sum_{u_1 \in U_1} \addch{f_{a,b}(u_1)}$ which can be done by splitting the sum on $U_1$ according to the value of $\mulch{u_1}$:
\begin{align*}
\sum_{u_1 \in U_1} \addch{f_{a,b}(u_1)}
& = \sum_{u_1 \in U_1} \addch{f_{a}(u_1)} \addch{g_b(u_1)} \\
& = \sum_{u_1\in U_1, b \mulch[m_0]{u_1} = 1} \addch{f_{a}(u_1)}
 - \sum_{u_1\in U_1, b \mulch[m_0]{u_1} \neq 1} \addch{f_{a}(u_1)} \\
& = 2 \sum_{u_1\in U_1, b \mulch[m_0]{u_1} = 1} \addch{f_{a}(u_1)}
 - \sum_{u_1\in U_1} \addch{f_{a}(u_1)} \enspace .
\end{align*}

As noted in Section~\ref{sec:kloo} the second sum is
\begin{align*}
\sum_{u_1 \in U_1} \addch{f_{a}(u_1)} & = 1 - K_{m_1}(a) \enspace .
\end{align*}

As far as the first one is concerned, let us denote it $\Snu[1]$.
As $m_1$ is odd,
using properties of the Dickson polynomial $D_3$ given in Section~\ref{sec:dickson},
one can show that for $b = 1$:
\begin{align*}
\Snu[1]
& = \frac{1}{3} \left( 1 - K_{m_1}(a) + 2 C_{m_1}(a, a) \right) \enspace .
\end{align*}
As $\Snu[1]$ takes the same value for both $b \neq 1$,
one deduces that for $b \neq 1$:
\begin{align*}
\Snu[1]
& = \frac{1}{3} \left( 1 - K_{m_1}(a) - C_{m_1}(a, a) \right) \enspace .
\end{align*}

Summarizing the above, we have the following proposition.
\begin{proposition}
For $\nu = 1$, $a \in \GF{m_1}^*$ and $b \in \GF[4]{}^*$,
the Walsh transform of $f_{a,b}$ at $\omega \in \GF{m_0}$ is,
for $\omega = 0$:
\begin{align}
\Wa{f_{a,b}}(0)
= \left\{
\begin{array}{ll}
1 + \frac{2^{m_1}-1}{3} \left( 1 - K_{m_1}(a) - 4 C_{m_1}(a, a) \right) & \text{if $b = 1$;} \\
1 + \frac{2^{m_1}-1}{3} \left( 1 - K_{m_1}(a) + 2 C_{m_1}(a, a) \right) & \text{if $b \neq 1$;}
\end{array}
\right.
\end{align}
and for $\omega \neq 0$:
\begin{align}
\Wa{f_{a,b}}(\omega)
& = \left\{
\begin{array}{ll}
1 + 2^{m_1} \addch{f_{a,b}(w_1^{-1})} + \frac{1}{3} \left( 1 - K_{m_1}(a) - 4 C_{m_1}(a, a) \right) & \text{if $b = 1$;}\\
1 + 2^{m_1} \addch{f_{a,b}(w_1^{-1})} + \frac{1}{3} \left( 1 - K_{m_1}(a) + 2 C_{m_1}(a, a) \right) & \text{if $b \neq 1$.}
\end{array}
\right.
\end{align}
\end{proposition}

The values of $C_{m_1}(a, a)$ were also computed by Carlitz~\cite{MR544577} when $m_1$ is odd,
and yield a concise and easy to compute the Walsh transform of $f_{a,b}$
at any $\omega \in \GF{m_0}$.

Together with Charpin \etal results~\cite{4595463,DBLP:journals/dm/CharpinHZ09}
and the Hasse--Weil bound on $K_{m_1}(a)$, these formulas prove that
$f_{a,b}$ is (hyper-)bent if and only if $K_{m_1}(a) = 4$
as was noted by Mesnager~\cite{DBLP:journals/dcc/Mesnager11}.
\begin{theorem}[\cite{DBLP:journals/dcc/Mesnager11}]
For $\nu = 1$, $a \in \GF{m}^*$ and $b \in \GF[4]{}^*$, the function $f_{a,b}$ is bent if and only if $K_{m_1}(a) = 4$.
\end{theorem}

\section{Even case}

\subsection{General extension degree}

In this section, it is supposed that $\nu \geq 2$, \ie both $m_0$ and $m_1$ are even.
The main difference with the case $\nu = 1$ is that $3$ does now divide $2^{m_1}-1$ (in fact $2^{m_\nu}+1$) rather than $2^{m_1}+1$,
 and $\mulch[m_0]{u}$ does not depend on the value of $u_1$ (but only on that of $u_\nu$).

In particular, the computation of $\sum_{u \in U} f_{a,b}(u)$ becomes straightforward:
\begin{align}
\sum_{u \in U} f_{a,b}(u)
& = \prod_{i=2}^{\nu-1} \left( 2^{m_i}+1 \right) \sum_{u_1 \in U_1} \addch{f_a(u_1)} \sum_{u_\nu \in U_\nu} \addch{g_b(u_\nu)} \nonumber \\
& = \prod_{i=2}^{\nu-1} \left( 2^{m_i}+1 \right) \frac{2^{m_\nu} + 1}{3} \left( 1 - K_{m_1}(a) \right) \sum_{c \in \GF[4]{}^*} \addch{\tr{2}{b c}} \nonumber \\
& = - \prod_{i=2}^{\nu-1} \left( 2^{m_i}+1 \right) \frac{2^{m_\nu} + 1}{3} \left( 1 - K_{m_1}(a) \right) \nonumber \\
& = - \frac{2^{2^{\nu-1} m_\nu} - 1}{3 \left(2^{m_\nu} - 1\right)} \left( 1 - K_{m_1}(a) \right) \enspace . \label{eqn:sumfab}
\end{align}

\subsection{A necessary condition}


The value of the Walsh transform at $\omega = 0$ given by Equation~\ref{eqn:walshzero}
can therefore be simplified.
\begin{lemma}
For $\nu > 1$, $a \in \GF{m_1}^*$ and $b \in \GF[4]{}^*$,
the Walsh transform of $f_{a,b}$ at $\omega = 0$ is
\begin{align}
\Wa{f_{a,b}}(0)
& = 1 - \frac{2^{m_1} - 1}{3} \left( 1 - K_{m_1}(a) \right) \enspace .
\end{align}
\end{lemma}

As noted by Mesnager~\cite{DBLP:journals/dcc/Mesnager11},
the Hasse--Weil bound on $K_{m_1}(a)$ implies that,
if $f_{a,b}$ is bent, then $\Wa{f_{a,b}}(0) = 2^{m_1}$
and $K_{m_1}(a) = 4$.
\begin{proposition}[\cite{DBLP:journals/dcc/Mesnager11}]
For $\nu > 1$, $a \in \GF{m_1}^*$ and $b \in \GF[4]{}^*$, if the function $f_{a,b}$ is bent, then $K_{m_1}(a) = 4$.
\end{proposition}

\subsection{Descending to an odd extension}

The value of the Walsh transform at $\omega \neq 0$ given by Equation~\ref{eqn:walshunit} becomes
\begin{align}
\Wa{f_{a,b}}(\omega)
& = 1 + \frac{2^{2^{\nu-1} m_\nu} - 1}{3 \left(2^{m_\nu} - 1\right)} \left( 1 - K_{m_1}(a) \right) \left( 1 - K_{m_1}(a) \right) + 2^{m_\nu} \sum_{u \in U, \tr[m_\nu]{m_0}{\omega u} = 0} \addch{f_{a,b}(u)} \enspace .
\end{align}
The sum over $u \in U$ can be split into smaller sums according to the extension $\GF{m_i}$
(with $1 \leq i \leq \nu$) where $\tr[m_i]{m_0}{u \omega}$ becomes $0$.

The first sum (corresponding to $i=1$) has to be dealt with separately and can be simplified as Equation~\ref{eqn:sumfab}:
\begin{align}
\sum_{u_1 = w_1^{-1}, u_2 \in U_2, \ldots, u_\nu \in U_\nu} \addch{f_a(u_1)} \addch{g_b(u_\nu)}
%& = - \left( 2^{m_2} + 1 \right) \cdots \left( 2^{m_{\nu-1}} + 1 \right) \frac{2^{m_\nu} + 1}{3} \addch{f_a(w_1^{-1})} \\
& = - \frac{2^{2^{\nu-1} m_\nu} - 1}{3\left(2^{m_\nu} - 1\right)} \addch{f_a(w_1^{-1})} \enspace . \label{eqn:sumfirst}
\end{align}

The second one (correspond to $i = 2$) is
\begin{align*}
\sum_{\substack{u_1 \neq w_1^{-1},\\ \tr[m_2]{m_0}{u_1 u_2 w_1 w_2} = 0,\\ u_3 \in U_3, \ldots, u_\nu \in U_\nu}} \addch{f_a(u_1)} \addch{g_b(u_\nu)}
& = \prod_{i=2}^{\nu-1} \left( 2^{m_i}+1 \right) \sum_{u_1 \neq w_1^{-1}} \addch{f_a(u_1)} \sum_{u_\nu \in U_\nu} \addch{g_b(u_\nu)} \\
& = - \frac{2^{2^{\nu-2} m_\nu} - 1}{3\left(2^{m_\nu} - 1\right)} \left( 1 - \addch{f_a(w_1^{-1})} - K_{m_1}(a) \right) \enspace ,
\end{align*}
together with similar sums:
\begin{align*}
\sum_{\substack{\tr[m_{i-1}]{m_0}{u_1 \cdots u_{i-1} w_1 \cdots w_{i-1}} \neq 0, \\ \tr[m_i]{m_0}{u_1 \cdots u_i w_1 \cdots w_i} = 0, \\u_{i+1} \in U_{i+1}, u_\nu \in U_\nu}} \hspace{-4em} \addch{f_a(u_1)} \addch{g_b(u_\nu)}
& = - \prod_{i=2}^{i-1} 2^{m_i}  \frac{2^{2^{\nu-2} m_\nu} - 1}{3\left(2^{m_\nu} - 1\right)} \left( 1 - \addch{f_a(w_1^{-1})} - K_{m_1}(a) \right) \enspace ,
\end{align*}
until the penultimate one (corresponding to $i = \nu - 1$):
\begin{align*}
\sum_{\substack{\tr[m_{\nu-2}]{m_0}{u_1 \cdots u_{\nu-2} w_1 \cdots w_{\nu-2}} \neq 0, \\ \tr[m_\nu-1]{m_0}{u_1 \cdots u_{\nu-1} w_1 \cdots w_{\nu-1}} = 0, \\ u_\nu \in U_\nu}} \hspace{-4em} \addch{f_a(u_1)} \addch{g_b(u_\nu)}
& = - \prod_{i=2}^{i-1} 2^{m_i} \frac{2^{m_\nu} + 1}{3} \left( 1 - \addch{f_a(w_1^{-1})} - K_{m_1}(a) \right) \enspace ;
\end{align*}
and they sum back up to
\begin{align}
%- \frac{2^{\left( 2^{\nu-2} - 1 \right) m_{\nu-1}}-1}{2^{m_{\nu-1}}-1} \frac{2^{m_\nu} + 1}{3} \left( 1 - \addch{f_a(w_1^{-1})} - K_{m_1}(a) \right)
& - \frac{2^{2 \left( 2^{\nu-2} - 1 \right) m_{\nu}}-1}{3 \left( 2^{m_{\nu}}-1 \right)} \left( 1 - \addch{f_a(w_1^{-1})} - K_{m_1}(a) \right) \label{eqn:summiddle}
\end{align}
as an easy induction shows.

The last sum (corresponding to $i = \nu$) can be split according to the value of
$\mulch[m_0]{u_\nu}$ as in Section~\ref{sec:odd}:
\begin{align}
\sum_{\substack{\tr[m_{\nu-1}]{m_0}{u \omega} \neq 0, \\ \tr[m_\nu]{m_0}{u \omega} = 0}} \addch{f_a(u_1)} \addch{g_b(u_\nu)}
& = 2 \sum_{\substack{\tr[m_{\nu-1}]{m_0}{u \omega} \neq 0, \\ \tr[m_\nu]{m_0}{u \omega} = 0,\\ b \mulch[m_0]{u_\nu} = 1}} \addch{f_a(u_1)}
 - \sum_{\substack{\tr[m_{\nu-1}]{m_0}{u \omega} \neq 0, \\ \tr[m_\nu]{m_0}{u \omega} = 0}} \addch{f_a(u_1)} \enspace , \label{eqn:sumlast}
\end{align}
where the second term is easily shown to be
\begin{align}
- \sum_{\substack{\tr[m_{\nu-1}]{m_0}{u \omega} \neq 0, \\ \tr[m_\nu]{m_0}{u \omega} = 0 }} \addch{f_a(u_1)}
%& = - 2^{m_2} \cdots 2^{m_{\nu - 1}} \left( 1 - \addch{f_a(w_1^{-1})} - K_{m_1}(a) \right) \\
& = - 2^{2 \left(2^{\nu - 2} - 1\right) m_\nu} \left( 1 - \addch{f_a(w_1^{-1})} - K_{m_1}(a) \right) \enspace . \label{eqn:sumeasy}
\end{align}

Equations~\ref{eqn:sumfirst}, \ref{eqn:summiddle}, \ref{eqn:sumlast} and \ref{eqn:sumeasy} lead to the following expression for the Walsh transform at $\omega \neq 0$:
\begin{align*}
\Wa{f_{a,b}}(\omega)
& = 1 + \frac{2^{2^{\nu-1} m_\nu} - 1}{3\left(2^{m_\nu} - 1\right)} \left( 1 - K_{m_1}(a) \right) \nonumber \\
& \qquad - 2^{m_\nu} \frac{2^{2^{\nu-1} m_\nu} - 1}{3\left(2^{m_\nu} - 1\right)} \addch{f_a(w_1^{-1})} \nonumber \\
& \qquad - 2^{m_\nu} \frac{2^{2 \left( 2^{\nu-2} - 1 \right) m_{\nu}}-1}{3 \left( 2^{m_{\nu}}-1 \right)} \left( 1 - \addch{f_a(w_1^{-1})} - K_{m_1}(a) \right) \nonumber \\
& \qquad - 2^{m_\nu} 2^{2 \left(2^{\nu - 2} - 1\right) m_\nu} \left( 1 - \addch{f_a(w_1^{-1})} - K_{m_1}(a) \right) \nonumber \\
%& \qquad + 2^{m_{\nu} + 1} \sum_{\substack{\tr[m_{\nu-1}]{m_0}{u \omega} \neq 0,\\ \tr[m_\nu]{m_0}{u \omega} = 0,\\ b \mulch[m_0]{u_\nu} = 1}} \addch{f_a(u_1)} \nonumber \\
& \qquad + 2^{m_{\nu} + 1} \sum_{\tr[m_{\nu-1}]{m_0}{u \omega} \neq 0, \tr[m_\nu]{m_0}{u \omega} = 0, b \mulch[m_0]{u_\nu} = 1} \addch{f_a(u_1)}\enspace ,
\end{align*}
which can be rewritten into a more compact form.
\begin{proposition}
\label{prp:snu}
For $\nu > 1$, $a \in \GF{m_1}^*$ and $b \in \GF[4]{}^*$,
and $\omega \in \GF{m_0}^*$, denote by $\Snu$ the sum
\begin{align}
%\sum_{\substack{\tr[m_{\nu-1}]{m_0}{u \omega} \neq 0,\\ \tr[m_\nu]{m_0}{u \omega} = 0,\\ b \mulch[m_0]{u_\nu} = 1}} \addch{f_a(u_1)} \label{eqn:gauss}
S_\nu(a, b, \omega) & = \sum_{\tr[m_{\nu-1}]{m_0}{u \omega} \neq 0, \tr[m_\nu]{m_0}{u \omega} = 0, b \mulch[m_0]{u_\nu} = 1} \addch{f_a(u_1)} \label{eqn:gauss} \enspace.
\end{align}
The Walsh transform of $f_{a,b}$ at $\omega \neq 0$ is
\begin{align}
\Wa{f_{a,b}}(\omega)
%& = 1 + \frac{2^{2^{\nu-1} m_\nu} - 1}{3\left(2^{m_\nu} - 1\right)} \left( 1 - 2^{m_\nu} \addch{f_a(w_1^{-1})} - K_{m_1}(a) \right) \nonumber \\
%& \qquad - \left( \frac{2^{\left( 2^{\nu-1} - 1 \right) m_{\nu}}-2^{m_\nu}}{3 \left( 2^{m_{\nu}}-1 \right)} + 2^{\left(2^{\nu - 1} - 1\right) m_\nu} \right) \left( 1 - \addch{f_a(w_1^{-1})} - K_{m_1}(a) \right) \nonumber \\
& = 1 - \frac{2 \cdot 2^{\left(2^{\nu-1}-1\right)m_\nu} - 1}{3} \left(1 - K_{m_1}(a)\right) \nonumber \\
& \qquad - \frac{2 \cdot 2^{\left( 2^{\nu-1} - 1 \right) m_\nu} \left( 2^{m_\nu - 1} - 1 \right)}{3} \addch{f_a(w_1)} \nonumber \\
& \qquad + 2^{m_{\nu} + 1} \Snu \enspace . \label{eqn:walshfull}
\end{align}
\end{proposition}

Unfortunately, making the remaining sum $S_\nu(a, b, \omega)$ explicit
is a hard problem.
Doing so is equivalent to evaluating a Gauss sum as in Equation~\ref{eqn:gengauss}:
an exponential sum involving a multiplicative character and an additive character.
In the next section, we manage to tackle the case $\nu = 2$
when $\omega \in \GF{m_1}^*$ (that is $w_1 = 1$)
and conjecture a partial formula when $\omega \not\in \GF{m_1}^*$.

\subsection{Four times an odd number}

From now on, it is supposed that $\nu = 2$, \ie $m_0$ is four times the odd number $m_2$.

For $\tr[m_2]{m_0}{u \omega}$ to be zero with $u_1 \neq w_1^{-1}$,
$u_2$ must be $u_2 = \left( \omega_2 \tr[m_1]{m_0}{u_1 \omega_1} \right)^{-1}$
so that the sum of Equation~\ref{eqn:gauss} becomes
\begin{align}
\label{eqn:fourgauss}
\Snu[2]
& = \sum_{u_1 \neq w_1^{-1}, \mulch[m_0]{w_2 \tr[m_1]{m_0}{u_1 w_1}} = b} \addch{\tr{m_0}{a u_1^{-2}}} \enspace .
\end{align}
%Remark that
%\begin{align*}
%b \left( w_2^{-1} \tr[m_1]{m_0}{u_1 w_1}^{-1} \right)^{\frac{2^{m_0}-1}{3}} & = b \left( w_2^{-1} \left( u_1 w_1 + u_1^{-1}w_1^{-1} \right)^{-1} \right)^{\frac{2^{m_0}-1}{3}} \\
%& = b w_2^{-\left( 2^{m_1} + 1 \right) \left( 2^{m_2} - 1 \right) \frac{2^{m_2}+1}{3}} \left( u_1 w_1 + u_1^{-1}w_1^{-1} \right)^{- \left( 2^{m_1} + 1 \right) \left( 2^{m_2} - 1 \right) \frac{2^{m_2}+1}{3}} \\
%& = b w_2^{4 \frac{2^{m_2}+1}{3}} \left( u_1 w_1 + u_1^{-1}w_1^{-1} \right)^{- 2 \frac{2^{m_1}-1}{3}} \\
%\end{align*}

\subsubsection{The subfield case}
We now restrict to the case $w_1 = 1$, that is $\omega \in \GF{m_1}^*$ rather than $\omega \in \GF{m_0}^*$ in full generality.

Define $\gamma \in \GF[4]{}^*$ by $\gamma = b \mulch[m_1]{w_2}$
%$\gamma = b w_2^{4 \frac{2^{m_2}+1}{3}}$,
and $\alpha \in \GF[4]{}^*$ by $\alpha = \mulch[m_1]{a}$.
Moreover, let $c \in \GF[2]{m_1}^*$ be such that $\mulch{c} = \gamma$.
and $\beta \in \GF[4]{}^*$ be a primitive third root of unity.
The sum of Equation~\ref{eqn:fourgauss} becomes
\begin{align}
\label{eqn:fourgaussone}
\Snu[2]
& =\sum_{u_1 \neq 1, \mulch[m_1]{u_1^2 + u_1^{-2}} = b \mulch[m_1]{w_2}} \addch{\tr{m_1}{a \left( u_1^{-2}+ u_1^{2} \right)}} \nonumber \\
& = \sum_{u_1 \neq 1, \mulch[m_1]{u_1 + u_1^{-1}} = b \mulch[m_1]{w_2}} \addch{\tr{m_1}{a \left( u_1 + u_1^{-1} \right)}} \nonumber \\
& = 2 \sum_{t \in \T_{m_1}^1, \mulch[m_1]{t} = b \mulch[m_1]{w_2}} \addch{\tr{m_1}{a t}} \nonumber \\
& = 2 \sum_{t \in \T_{m_1}^1, \mulch[m_1]{t} = \gamma} \addch{\tr{m_1}{a t}} \enspace .
\end{align}
The final sum in Equation~\ref{eqn:fourgaussone} can be seen as a first step
toward generalizing the sum computed in Section~\ref{sec:odd} in the odd case:
rather than involving $u_1$ directly, it involves its trace $t = \tr[m_1]{m_0}{u_1}$.

%First notice that:
%\begin{align*}
%2 \sum_{t \in \T_{m_1}^1} \addch{\tr{m_1}{a t}} = - K_{m_1}(a) \\
%\end{align*}

The computation can then be reduced to that of sums over all of $\GF{m_1}^*$:
\begin{align*}
2 \sum_{t \in \T_{m_1}^1, \mulch[m_1]{t} = \gamma} \addch{\tr{m_1}{a t}}
& = \sum_{x \in \GF{m_1}^*, \mulch[m_1]{x} = \gamma} \addch{\tr{m_1}{a x}} \\
& \qquad - \sum_{x \in \GF{m_1}^*, \mulch[m_1]{x} = \gamma} \addch{\tr{m_1}{a x + x^{-1}}} \enspace .
\end{align*}

The first sum is easily seen to be a cubic sum:
\begin{align*}
\sum_{x \in \GF{m_1}^*, \mulch{x} = \gamma} \addch{\tr{m_1}{a x}}
& = \sum_{x \in \GF{m_1}^*, \mulch{x} = \mulch{c}} \addch{\tr{m_1}{a x}} \\
& = \sum_{x \in \GF{m_1}^*, \mulch{x} = 1} \addch{\tr{m_1}{a c x}} \\
& = \frac{1}{3} \sum_{x \in \GF{m_1}^*} \addch{\tr{m_1}{a c x^3}} \\
& = \frac{1}{3} C_{m_1}(a c, 0) \enspace .
\end{align*}
As $m_1$ is even and $m_2$ is odd, Carlitz's results~\cite{MR544577} imply
\begin{align}
\sum_{x \in \GF{m_1}^*, \mulch{x} = \gamma} \addch{\tr{m_1}{a x}}
& = \left\{
\begin{array}{ll}
\frac{2^{m_2+1} - 1}{3} & \text{if $\gamma = \alpha^{-1}$;} \\
-\frac{2^{m_2} + 1}{3} & \text{if $\gamma \neq \alpha^{-1}$.}
\end{array}
\right. \label{eqn:twistedcubic}
\end{align}
%In particular, its value only depends on $\alpha \gamma$.

Let us now proceed with the second sum.
First remark that summing over the three possible values of $\gamma$ yields
\begin{align*}
\sum_{x \in \GF{m_1}^*} \addch{\tr{m_1}{a x + x^{-1}}}
& = K_{m_1}(a) - 1 \enspace .
\end{align*}
Moreover, making the change of variable $x = \left(a x\right)^{-1}$ shows that
the sum takes the same value at $\gamma$ and $\alpha^{-1} \gamma^{-1} = \alpha^2 \gamma^2$.
For example, if $\alpha = 1$, that is if $a$ is a cube,
it means it takes the same value for $\gamma \in \{ \beta, \beta^2 \}$.
In any case, it is sufficient to determine the value of the sum for one value
of $\gamma$ to deduce the values for all $\gamma$'s.

Denote by $r$ a square root of $a$.
The change of variable $x = rx$ and properties of the Dickson polynomial
$D_3$when $m_1$ is even show that for $\gamma = \alpha = \mulch[m_1]{r^{-1}}$:
\begin{align}
\sum_{x \in \GF{m_1}^*, \mulch{x} = \alpha} \addch{\tr{m_1}{a x + x^{-1}}}
& = \sum_{x \in \GF{m_1}^*, \mulch{x} = 1} \addch{\tr{m_1}{r \left( x + x^{-1} \right)}} \nonumber \\
& = \frac{1}{3} \sum_{x \in \GF{m_1}^*} \addch{\tr{m_1}{r \left( x^3 + x^{-3} \right)}} \nonumber \\
& = \frac{1}{3} \sum_{x \in \GF{m_1}^*} \addch{\tr{m_1}{r D_3(x + x^{-1})}} \nonumber \\
%& = \frac{1}{3} \left(2 \sum_{t \in \T_{m_1}^0 \setminus \set{0, 1}} \addch{\tr{m_1}{r D_3(t)}} + 1\right) \nonumber \\
& = \frac{1}{3} \left(2 \sum_{t \in \T_{m_1}^0} \addch{\tr{m_1}{r D_3(t)}} - 1\right) \nonumber \\
%& = \frac{1}{3} \left(2 \left( C_{m_1}(r, r) - \sum_{t \in \T_{m_1}^1} \addch{\tr{m_1}{r D_3(t)}} \right) - 1\right) \nonumber \\
& = \frac{1}{3} \left(2 C_{m_1}(r, r) - 2 \sum_{t \in \T_{m_1}^1} \addch{\tr{m_1}{r t}} - 1\right) \nonumber \\
& = \frac{1}{3} \left(2 C_{m_1}(r, r) + K_{m_1}(r) - 1\right) \nonumber \\
& = \frac{1}{3} \left(2 C_{m_1}(a, a) + K_{m_1}(a) - 1\right) \enspace . \label{eqn:twistedklooeq}
\end{align}

For $\gamma \neq \alpha$, one deduces:
\begin{align}
\sum_{x \in \GF{m_1}^*, \mulch{x} = \gamma} \addch{\tr{m_1}{a x + x^{-1}}}
& = \frac{1}{3} \left(- C_{m_1}(a, a) + K_{m_1}(a) - 1\right) \enspace . \label{eqn:twistedklooneq}
\end{align}

Gathering Equations~\ref{eqn:twistedcubic}, \ref{eqn:twistedklooeq} and
\ref{eqn:twistedklooneq} gives the following expressions for $\Snu[2]$.
\begin{proposition}
For $\nu = 2$, $a \in \GF{m_1}^*$ and $b \in \GF[4]{}^*$,
and $\omega \in \GF{m_1}^*$,
let $\gamma = b \mulch[m_1]{w_2}$.
Then the sum $\Snu[2]$ is
\begin{align}
\Snu[2]
& = \left\{
\begin{array}{ll}
\frac{2^{m_2+1} - 1}{3} - \frac{1}{3} \left(2 C_{m_1}(a, a) + K_{m_1}(a) - 1\right) & \text{if $\gamma = \alpha = 1$;}\\
-\frac{2^{m_2} + 1}{3} - \frac{1}{3} \left(2 C_{m_1}(a, a) + K_{m_1}(a) - 1\right) & \text{if $\gamma = \alpha \neq 1$;}\\
\frac{2^{m_2+1} - 1}{3} - \frac{1}{3} \left(- C_{m_1}(a, a) + K_{m_1}(a) - 1\right) & \text{if $\gamma = \alpha^{-1} \neq 1$;}\\
-\frac{2^{m_2} + 1}{3} - \frac{1}{3} \left(- C_{m_1}(a, a) + K_{m_1}(a) - 1\right) & \text{if $\gamma \neq \alpha$ and $\gamma \neq \alpha^{-1}$.}
\end{array}
\right.
\end{align}
\end{proposition}

Carlitz's results~\cite{MR544577} recalled in Section~\ref{sec:cubic}
can be used to make the cubic sum $C_{m_1}(a, a)$ explicit.
In the particular case where $K_{m_1}(a) \equiv 1 \pmod{3}$,
which is equivalent to $C_{m_1}(a, a) = 0$ and implies that $a$ is a cube,
the expression for $\Snu[2]$ gets very concise,
as does Equation~\ref{eqn:walshfull} for the Walsh transform.
\begin{corollary}
\label{crl:walshfour}
For $\nu = 2$, $a \in \GF{m_1}^*$ with $K_{m_1}(a) \equiv 1 \pmod{3}$
and $b \in \GF[4]{}^*$, and $\omega \in \GF{m_1}^*$,
let $\gamma = b \mulch[m_1]{w_2}$.
Then the sum $\Snu[2]$ is
\begin{align}
\Snu[2]
%& = \left\{
%\begin{array}{ll}
%\frac{2^{m_2+1} - K_{m_1}(a)}{3} & \text{if $\gamma = 1$;}\\
%- \frac{2^{m_2} + K_{m_1}(a)}{3} & \text{if $\gamma \neq 1$;}
%\end{array}
%\right. \nonumber \\
& = \frac{2^{m_2+1} - K_{m_1}(a)}{3} - \left( 1 - \addch{\tr{2}{\gamma}} \right) 2^{m_2-1} \enspace .
\end{align}
and the Walsh transform at $\omega \neq 0$ is
\begin{align}
\Wa{f_{a,b}}(\omega)
& = 2^{m_1} \addch{\tr{2}{\gamma}} + \frac{4 - K_{m_1}(a)}{3} \enspace .
\end{align}

\end{corollary}
%\begin{proof}
%\begin{align*}
%\Wa{f_{a,b}}(\omega)
%& = 1 - \frac{2^{m_2 + 1} - 1}{3} \left(1 - K_{m_1}(a)\right)
% - \frac{2^{m_2 + 1} \left( 2^{m_2 - 1} - 1 \right)}{3} \\
%& \qquad + 2^{m_2 + 1} \left( \frac{2^{m_2+1} - K_{m_1}(a)}{3} - \left( 1 - \addch{\tr{2}{\gamma}} \right) 2%^{m_2-1} \right) \enspace .
%\end{align*}
%\end{proof}
Note that Corollary~\ref{crl:walshsubfield} shows that
for $\omega \in \GF{m_1}^*$ the Walsh transform of $f_{a,b}$
is that of a bent function.

\subsection{A conjectural general formula}

The techniques used in the previous section do not apply to the general case
where $w_1 \neq 1$, \ie $\omega \in \GF{m_0}^*$.
The main reason being that the multiplicative and additive characters
of $\GF{m_1}$ act on different values, \eg
$r = \tr[m_1]{m_0}{w_1 u_1}$, or $s = \tr[m_1]{m_0}{w_1^{-1} u_1}$,
and $t = \tr[m_1]{m_0}{u_1}$.
Considering $v = \tr[m_1]{m_0}{w_1}$, these values are related by $r + s = v t$.
Moreover, the sum $\Snu[2]$ takes the same value for $w_1$ and $w_1^{-1}$,
so there is hope to introduce enough symmetry to reduce the case $w_1 \neq 1$
to the case $w_1 = 1$ using elementary techniques.
Unfortunately we could not devise a way to do so.

Yet, experimental evidence suggests the following conjecture
which relates the value of the Walsh transform at $w_1 = 1$ and $w_1 \neq 1$.
\begin{conjecture}
\label{cnj:walshconj}
For $\nu = 2$, $a \in \GF{m_1}^*$ with $K_{m_1}(a) \equiv 1 \pmod{3}$
and $b \in \GF[4]{}^*$, and $\omega \in \GF{m_0}^*$,
let $\gamma = b \mulch[m_1]{w_2}$.
Then there exists a Boolean function $h(a, b, w_1, w_2)$ such that
the sum $\Snu[2]$ is
\begin{align*}
\Snu[2]
& = \frac{2^{m_2+1} - K_{m_1}(a)}{3} \nonumber \\
& \qquad - \left( 1 - \addch{\mystery} \right) 2^{m_2-1} \nonumber \\
& \qquad - \left( 1 - \addch{\tr{m_0}{a w_1^2}} \right) \frac{2^{m_2-1} - 1}{3} \enspace .
\end{align*}
The Walsh transform at $\omega \neq 0$ is then
\begin{align}
\Wa{f_{a,b}}(\omega)
& = 2^{m_1} \addch{\mystery} + \frac{4 - K_{m_1}(a)}{3} \enspace .
\end{align}
\end{conjecture}
In particular, this conjecture implies Conjecture~\ref{cnj:kloofour}:
if $K_{m_1}(a) = 4$, then $f_{a,b}$ is bent.
(And Corollary~\ref{crl:walshsubfield} already does so
when $\omega \in \GF{m_1}^*$.)

\subsection{Experimental data}

The computation of $\Snu[2]$ was implemented in C and assembly\footnote{%
The source code is available at
\url{https://github.com/jpflori/expsums}.},
using AVX extensions for the arithmetic of $\GF{m_0}^*$,
PARI/GP~\cite{PARI2} to compute the Kloosterman sums $K_{m_1}(a)$,
and OpenMP~\cite{openmp} for parallelization.
The computational cost of verifying Conjecture~\ref{cnj:walshconj}
can be somewhat leveraged using elementary properties of $\Snu[2]$:
\begin{itemize}
\item it only depends on the cyclotomic class of $a \in \GF{m_1}^*$,
\item it is the same for $w_1$ and $w_1^-1$;
\item the inner value can be computed at the same time for $u_1$ and $u_1^{-1}$.
\end{itemize}
Whatsoever, there are:
\begin{itemize}
\item $3$ values of $\gamma \in \GF[4]{}^*$;
\item $\tilde{O}(2^{m_1})$ values of $a \in \GF{m_1}^*$ to check;
\item $2^{m_1-1}$ values of $w_1 \in U_1 \backslash \set{1}$;
\item $\tilde{O}(2^{m_1})$ operations in $\GF{m_0}$ for each $w_1$.
\end{itemize}
Therefore, checking the conjectured formula for $\Snu[2]$ over $\GF{m_0}$
has time complexity $\tilde{O}(2^{3 m_1})$ which quickly becomes overcostly
(and is comparable to that of computing the Walsh spectrum
for every cyclotomic class of $a \in \GF{m_1}^*$ which has time complexity
$\tilde{O}(2^{3 m_1})$ as well but space complexity $\tilde{O}(2^{m_1})$).
Still, we checked Conjecture~\ref{cnj:walshconj}
\begin{itemize}
\item completely for $m_2 = 3, 5, 7, 9$,
\item for $i$ up to $3405$ where $a = z^i$
and $z$ is a primitive element of $\GF{m_1}$ for $m_2 = 11$.
\end{itemize}

Finally, assuming $K_{m_1}(a) \equiv 1 \pmod{3}$
and Conjecture~\ref{cnj:walshconj} is correct,
Parseval's equality yields the following relation:
\begin{align*}
\sum_{x \in \GF{m_0}^*} \addch{\mystery}
& = \frac{2^{m_1} - 1}{3} \left( K_{m_1}(a) - 1 \right) \\
& = \Wa{f_{a,b}}(0) - 1 \enspace .
\end{align*}
This is supported by experimental evidence that
there are exactly $2^{m_1 - 1} + (5/6) \left(K_{m_1}(a) - 4\right) + 3$
(respectively $2^{m_1 - 1} - (1/6)\left(K_{m_1}(a) - 4\right)$) values of $w_1 \in U_1$
such that $\mystery$ is zero when $\gamma = 1$
(respectively $\gamma \neq 1$).

\section{Further research and open problems}

Hopefully, Conjecture~\ref{cnj:walshconj} can be proved using only elementary
techniques as the ones used by Mesnager~\cite{DBLP:journals/dcc/Mesnager11}
and in this note.
Otherwise, more involved techniques could be tried, \eg considering a whole
family of sums as a whole and their geometric structure.
Another posibility would be to directly treat the general Gauss sums of
Equations~\ref{eqn:gengauss} and~\ref{eqn:gauss} without focussing
on the case $\nu = 2$.

\bibliographystyle{plain}
\bibliography{even}

\end{document}
