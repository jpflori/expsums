\documentclass[a4paper]{article}

\usepackage[english]{babel}
\usepackage[utf8]{inputenc}
\usepackage{lmodern}
\usepackage[T1]{fontenc}
\usepackage{datetime}
\usepackage{xifthen}

\usepackage{amsmath,amssymb}
\usepackage{mathtools}
\usepackage{geometry}
\usepackage{hyperref}
\usepackage{microtype}
\usepackage{verbatim}
\usepackage{rotating}
\usepackage{array}
\usepackage[ruled,linesnumbered]{algorithm2e}
\usepackage{tikz}
\usepackage{pgfplots}

\usepackage{xspace}
\newcommand{\ie}{i.e.\@\xspace}
\newcommand{\eg}{e.g.\@\xspace}
\newcommand{\etal}{et al.\@\xspace}

\usepackage{amsthm}
\newtheorem{theorem}{Theorem}[section]
\newtheorem{definition}[theorem]{Definition}
\newtheorem{proposition}[theorem]{Proposition}
\newtheorem{remark}[theorem]{Remark}
\newtheorem{example}[theorem]{Example}
\newtheorem{corollary}[theorem]{Corollary}
\newtheorem{lemma}[theorem]{Lemma}
\newtheorem{conjecture}[theorem]{Conjecture}

\newcommand{\N}{\mathbb{N}}
\newcommand{\Z}{\mathbb{Z}}
\newcommand{\Q}{\mathbb{Q}}
\newcommand{\GF}[2][2]{\mathbb{F}_{#1^{#2}}}
\newcommand{\T}{\mathcal{T}}
\newcommand{\set}[1]{\left\{ #1 \right\}}
\newcommand{\card}[1]{\left| #1 \right|}
\DeclareMathOperator{\lcm}{lcm}
\DeclareMathOperator{\Tr}{Tr}
\makeatletter
\newcommand{\tr}[3][1]{\ifthenelse{\isempty{#3}}%
  {\Tr_{#1}^{#2}}%
  {\Tr_{#1}^{#2}\left(#3\right)}}
\newcommand{\addch}[1]{\ifthenelse{\isempty{#1}}%
  {\chi}%
  {\chi \left( #1 \right)}}
\newcommand{\mulch}[1]{\ifthenelse{\isempty{#1}}%
  {\psi}%
  {\psi \left( #1 \right)}}
\makeatother
\newcommand{\Wa}[1]{\widehat{\chi_{#1}}}

\hypersetup{pdftitle={A conjecture about Gauss sums and bentness of binomial Boolean functions},%
  pdfauthor={Jean-Pierre Flori},%
  pdfsubject={Mathematics and Cryptography},%
  pdfcreator={Jean-Pierre Flori},%
  pdfproducer={Jean-Pierre Flori},%
  pdfkeywords={mathematics} {cryptography} {Boolean functions} {bent functions} {Walsh spectrum} {Kloosterman sums},
}

\title{A conjecture about Gauss sums and bentness of binomial Boolean functions}

\author{Jean-Pierre Flori
  \thanks{ANSSI (Agence nationale de la sécurité des systèmes d'information),
    51, boulevard de La Tour-Maubourg,
    75700 Paris 07 SP, France.
    \texttt{jean-pierre.flori@ssi.gouv.fr}}
}

\date{\today~--~\currenttime}
% \date{\today}

\begin{document}

\maketitle

\begin{abstract}
  In this note, the polar decomposition of binary fields of even extension degree is
  used to reduce the evaluation of the Walsh transform of binomial Boolean functions
  to that of Gauss sums.
  In the case of extensions of degree four times an odd number, an explicit formula involving a Kloosterman sum is conjectured,
  proved with further restrictions, and supported by extensive experiments when
  no further hypotheses are made.
  In particular, the validity of this formula is shown to be equivalent
  to a simple and efficient characterization for bentness
  previously conjectured by Mesnager.
\end{abstract}

\noindent
{\bf Keywords}. Boolean functions, bent functions, Walsh spectrum, exponential sums, Gauss sums, Kloosterman sums.


\section{Introduction}
\label{sec:introduction}

Bent functions are Boolean functions defined over an extension of even
degree and achieving optimal non-linearity.
They are of both combinatorial and cryptographic interest.
Unfortunately, characterizing bentness is a difficult problem,
and the less general question of providing simple and efficient
criteria within infinite families of functions in a specific polynomial form
is still challenging.

For a Boolean function $f$ defined over $\GF{n}$ with $n = 2 m$ and
given in polynomial form, a simple characterization for bentness
is that its Walsh transform $\Wa{f}$ values are only $2^{\pm m}$.
Nevertheless, such a characterization is neither concise nor efficient:
the best algorithm to compute the full Walsh spectrum has complexity $O(n 2^n)$
and this is asymptotically optimal, whence the need to restrict to functions
in a given form and to look for more efficient criteria.
Unfortunately, only a few infinite families of Boolean functions
with a simple and efficient criterion for bentness are known.

The most classical family is due to Dillon~\cite{MR2624542}
and is made of monomial functions:
\[
f_a(x) = \tr{n}{a x^{r(2^m-1)}} \enspace ,
\]
where $n = 2 m$, $a \in \GF{n}$ and $r$ is co-prime with $2^m + 1$.
Such functions are bent (and even hyper-bent) if and only if the Kloosterman sum $K_m(a)$
associated with $a$ is equal to zero~\cite{MR2624542,DBLP:journals/tit/Leander06,DBLP:journals/tit/CharpinG08}.
Not only does such a criterion gives a concise and elegant characterization for bentness,
but using the connection between Kloosterman sums and elliptic curves~\cite{MR0308088,MR925289,MR1054286}
it also allows to check for it in polynomial time~\cite{DBLP:conf/seta/Lisonek08,DBLP:journals/corr/abs-1104-3882}.
Further results on Kloosterman sums involving $p$-adic arithmetic~\cite{MR2794931,6126036,Moloney:PHD}
lead to even faster generation of zeros of Kloosterman sums and so of (hyper-)bent functions.

Mesnager~\cite{DBLP:journals/dcc/Mesnager11} proved a similar criterion
for a family Boolean functions in binomial form:
\[
f_{a,b}(x) = \tr{n}{a x^{r(2^m-1)}} + \tr{2}{b x^{\frac{2^n-1}{3}}} \enspace ,
\]
where $n = 2 m$, $a \in \GF{n}$, $b \in \GF[4]{}^*$
and $r$ is co-prime with $2^m + 1$
(but also $r = 3$ which divides $2^m+1$~\cite{DBLP:conf/ima/Mesnager09}).
When the extension degree $n$ is twice an odd number, that is when $m$ is odd,
$f_{a,b}$ is (hyper-)bent if and only if $K_m(a) = 4$.
Unfortunately, the proof does not extend to the case where $m$ is even.
Nonetheless, it is easy to show that $K_m(a) = 4$ is still a necessary
condition for $f_{a,b}$ to be bent in this latter case
(but note that $f_{a,b}$ can not hyper-bent in that case)
and it was conjectured to be sufficient condition.
Further experimental evidence gathered by Flori, Mesnager
and Cohen~\cite{DBLP:journals/iacr/FloriMC11} supported this conjecture:
for $m$ up to $16$, $f_{a,b}$ is bent if and only if $K_m(a) = 4$.

In this note, the polar decomposition of fields of even extension degree
$n = 2^\nu m$ is used to reduce the evaluation of the Walsh transform
of $f_{a,b}$ at $\omega \in \GF{n}^*$ to that of a Gauss sum of the form
\begin{align*}
\sum_{u \in U} \mulch{b \tr[m]{n}{\omega u}} \addch{\tr{n}{a u^{2^{2^{\nu-1}m}-1}}} \enspace ,
\end{align*}
where $\GF{n}^*$ is decomposed as $\GF{n}^* \simeq U \times \GF{m}^*$,
$\mulch{}$ is a cubic multiplicative character
and $\addch{}$ a quadratic additive character.
In the case of extensions of degree four times an odd number,
that is when $m$ is twice an odd number,
an explicit formula involving the Kloosterman sum $K_m(a)$ is partly
proved conjectured
and supported by extensive experiments.
In particular, the validity of this formula would prove the characterization
conjectured by Mesnager for extensions of degree four times an odd number.

\section{Notation}

Let $n \geq 2$ be an even integer and
$\nu \geq 1$ denote its $2$-valuation.
We denote by $m_i$ for $0 \leq i \leq \nu$ the integer $n / 2^i$,
\eg $m_0 = n$ and $m_\nu = m$ with the notation of the previous section.

For $0 \leq i < \nu$,  the extension degree of $\GF{m_i}$ is even and
it multiplicative group  $\GF{m_i}^*$ can be split using the so-called
polar decomposition
\[
\GF{m_i}^* \simeq U_{i+1} \times \GF{m_{i+1}}^* \enspace ,
\]
where $U_{i+1} \subset \GF{m_i}^*$ is the subgroup of $2^{m_{i+1}}+1$-th roots of unity
and $\GF{m_{i+1}}^*$ the subgroup of $2^{m_{i+1}}-1$-th roots of unity.
Therefore, $\GF{m_0}^*$ can be recursively decomposed as
\[
\GF{m_0}^* \simeq U_1 \times \cdots \times U_\nu \times \GF{m_\nu}^* \enspace .
\]
Let $U$ denote the image of $U_1 \times \cdots \times U_\nu$ within $\GF{m_0}^*$,
that is the set of $(2^{m_1}+1) \cdots (2^{m_\nu}+1)$-th roots of unity.
Then $\GF{m_0}^*$ can be decomposed as
\[
\GF{m_0}^* \simeq U \times \GF{m_\nu}^* \enspace .
\]
Remark that $3$ divides $2^{m_\nu}+1$ and is coprime with $2^{m_\nu}+1$ and $2^{m_i}+1$ for $0 \leq i < \nu$.

%For extension degrees $k$ and $l$ such that $k$ divides $l$,
%the field trace from $\GF{l}$ down to $\GF{k}$ is denoted by $\tr[k]{l}{x}$.

The non-principal additive character of $\GF{}$ is denoted $\addch{}$
and the Walsh transform of a Boolean function $f$ at $\omega \in \GF{m_0}$ is
\begin{align*}
\Wa{f}(\omega) & = \sum_{x \in \GF{m_0}} \addch{f(x) + \tr{m_0}{\omega x}} \enspace ,
\end{align*}
and $f$ is bent if and only if $\Wa{f}(\omega) = 2^{\pm m_1}$ for all $\omega \in \GF{m_0}$.

The Kloosterman sum of $a \in \GF{m_1}$ is
\[
K_{m_1}(a) = \sum_{x \in \GF{m_1}} \addch{\tr{m_1}{a x + \frac{1}{x}}} \enspace .
\]

The non-principal cubic multiplicative character $\mulch{}$ is defined as
\[
\mulch{x} = x^{\frac{2^{m_0}-1}{3}} \enspace ,
\]
for $x \in \GF{m_0}$.

The Boolean functions $f_{a,b}$ studied in this note are defined over $\GF{m_0}$
and given in binomial form:
\begin{align*}
f_{a,b}(x) & = \tr{m_0}{a x^{2^{m_1}-1}} + \tr{2}{b x^{\frac{2^{m_0}-1}{3}}} \\
& = \tr{m_0}{a x^{2^{m_1}-1}} + \tr{2}{b \mulch{x}} \enspace ,
\end{align*}
where $a \in \GF{m_0}$ and $b \in \GF[4]{}^*$.
We also define $f_a = f_{a,0}$ (corresponding to Dillon's monomial) and
$g_b(x) = \tr{2}{b \mulch{x}}$.

\section{Preliminaries}

\subsection{Field of definition of the coefficients}

First notice that it is enough to know how to evaluate the Walsh transform of
functions $f_{a,b}$ for $a \in \GF{m_1}^*$ rather than $\GF{m_0}^*$.
\begin{lemma}
Let $a \in \GF{m_0}^*$ be written as $a = \alpha \tilde{a}$
with $\alpha \in U_1$ and $\tilde{a} \in \GF{m_1}^*$
using the polar decomposition of $\GF{m_0}^*$.
Let $\tilde{\alpha} \in U_1$ be a squareroot of $\alpha$
and $\beta \in \GF[4]{}^*$ be $\beta = \tilde{\alpha}^{(2^{m_0}-1)/3}$.
Then
\begin{align*}
\Wa{f_{a,b}}(\omega) & = \Wa{f_{\tilde{a},\beta b}}(\tilde{\alpha} \omega) \enspace .
\end{align*}
\end{lemma}
\begin{proof}
Indeed, $x \mapsto \tilde{\alpha} x$ induces a permutation of $\GF{m_0}$,
and $\tilde{\alpha}^{2^{m_1}-1} = \tilde{\alpha}^{-2} = \alpha^{-1}$,
so that
\begin{align*}
\Wa{f_{a,b}}(\omega) & = \sum_{x \in \GF{m_0}} \addch{f_{a,b}(x) + \tr{m_0}{\omega x}} \\
& = \sum_{x \in \GF{m_0}} \addch{f_{a,b}(\tilde{\alpha} x) + \tr{m_0}{\omega \tilde{\alpha} x}} \\
& = \sum_{x \in \GF{m_0}} \addch{f_{\tilde{a},\beta b}(x) + \tr{m_0}{\omega \tilde{\alpha} x}} \\
& = \Wa{f_{\tilde{a},\beta b}}(\tilde{\alpha} \omega) \enspace . \qedhere
\end{align*}
\end{proof}
From now on we can suppose that $a \in \GF{m_1}^*$ without loss of generality.

\subsection{Polar decomposition}

We now start splitting the Walsh transform of $f_{a,b}$ at $\omega \in \GF{m_0}$ using the polar decomposition of $\GF{m_0}^*$ as $\GF{m_0}^* = U_1 \times \cdots \times U_\nu \times \GF{m_\nu}^* = U \times \GF{m_\nu}^*$.
We write $x \in \GF{m_0}^*$ as $x = u_1 \cdots u_\nu y = u y$ for $u_i \in U_i$, $u = u_1 \cdots u_\nu \in U$, and $y \in \GF{m_\nu}^*$.
The Walsh transform of $f_{a,b}$ at $\omega \in \GF{m_0}$ can then be written
\begin{align*}
\Wa{f_{a,b}}(\omega) & = \sum_{x \in \GF{m_0}} \addch{f_{a,b}(x) + \tr{m_0}{\omega x}} \\
& = 1 + \sum_{x \in \GF{m_0}^*} \addch{f_{a,b}(x) + \tr{m_0}{\omega x}} \\
& = 1 + \sum_{(u, y) \in U \times \GF{m_\nu}^*} \addch{f_{a,b}(u y)} \addch{\tr{m_0}{\omega u y}} \enspace .
\end{align*}
Note that $3$ divides $2^{m_\nu}+1$ so that $\frac{2^{m_0}-1}{3} = (2^{m_\nu}-1) \frac{2^{m_\nu}+1}{3} \prod_{i=1}^{\nu-1}(2^{m_i}+1)$ and $f_{a,b}(u y) = f_{a,b}(u)$.
Therefore
\begin{align*}
\Wa{f_{a,b}}(\omega) & = 1 + \sum_{u \in U} \addch{f_{a,b}(u)} \sum_{y \in \GF{m_\nu}^*} \addch{\tr{m_\nu}{\tr[m_\nu]{m_0}{\omega u} y}} \enspace .
\end{align*}
The sum ranging over $\GF{m_\nu}^*$ is equal to $-1$ when $\tr[m_\nu]{m_0}{\omega u} \neq 0$ and $2^{m_1}-1$ when $\tr[m_\nu]{m_0}{\omega u} = 0$.

When $\omega = 0$, the Walsh transform is
\begin{align}
\Wa{f_{a,b}}(0) & = 1 + \left( 2^{m_\nu} - 1 \right) \sum_{u \in U} \addch{f_{a,b}(u)} \enspace . \label{eqn:walshzero}
\end{align}

When $\omega \in \GF{m_0}^*$, write its polar decomposition $\omega = w_1 \cdots w_\nu o = w o$ where $(w_1, \ldots, w_\nu, o) \in U_1 \times \cdots U_\nu \times \GF{m_\nu}^*$ and $w = w_1 \cdots w_\nu$.
Then $\tr[m_\nu]{m_0}{\omega u} = 0$ if and only if there exists $1 \leq i \leq \nu$
such that $u_i = w_i^{-1}$.
The Walsh transform is then
\begin{align}
\Wa{f_{a,b}}(\omega) & = 1 - \sum_{u \in U, \tr[m_\nu]{m_0}{\omega u} \neq 0} \addch{f_{a,b}(u)} + (2^{m_\nu} - 1) \sum_{u \in U, \tr[m_\nu]{m_0}{\omega u} = 0} \addch{f_{a,b}(u)} \nonumber \\
& = 1 - \sum_{u \in U} \addch{f_{a,b}(u)} + 2^{m_\nu} \sum_{u \in U, \tr[m_\nu]{m_0}{\omega u} = 0} \addch{f_{a,b}(u)} \enspace . \label{eqn:walshunit}
\end{align}
To go further, the cases $\nu = 1$ and $\nu \geq 2$ have to be dealt with separately.

\section{Odd case}
\label{sec:odd}

In this section, it is supposed that $\nu = 1$, \ie $m_1$ is odd and $U = U_1$, which is the case that Mesnager settled~\cite{DBLP:journals/dcc/Mesnager11} and that is recalled here for completeness and because it inspired the treatment of the general case.

For $\omega \neq 0$, $\tr[m_1]{m_0}{\omega u_1} = 0$ if and only if $u_1 = w_1^{-1}$, so that
\begin{align*}
\sum_{u_1 \in U_1, \tr[m_1]{m_0}{\omega u_1} = 0} \addch{f_{a,b}(u_1)}
& = \addch{f_{a,b}(w_1^{-1})} \enspace .
\end{align*}

The only difficulty lies in the computation of $\sum_{u_1 \in U_1} \addch{f_{a,b}(u_1)}$ which can be done by splitting the sum on $U_1$ according to the value of $\mulch{u_1}$:
\begin{align*}
\sum_{u_1 \in U_1} \addch{f_{a,b}(u_1)} & = \sum_{u_1 \in U_1} \addch{f_{a}(u_1)} \addch{\tr{2}{b \mulch{u_1}}} \\
& = \sum_{u_1\in U_1, b \mulch{u_1} = 1} \addch{f_{a}(u_1)} - \sum_{u_1\in U_1, b \mulch{u_1} \neq 1} \addch{f_{a}(u_1)} \\
& = 2 \sum_{u_1\in U_1, b \mulch{u_1} = 1} \addch{f_{a}(u_1)} - \sum_{u_1\in U_1} \addch{f_{a}(u_1)} \enspace .
\end{align*}

It is well known that the second sum is
\begin{align*}
\sum_{u_1 \in U_1} \addch{f_{a}(u_1)} & = 1 - K_{m_1}(a) \enspace .
\end{align*}

As far as the first one is concerned, it is enough to compute it for $b = 1$.
Indeed, set $\beta$ to be a primitive third root of unity (which generates $\GF[4]{}^*$),
then it takes the same value for $b = \beta$ and $b = \beta^2$,
and the sum over these three values has just been computed.

Let $D_3(x) = x^3 + x$ be the third Dickson polynomial of the first kind and $\T_{m_1}^i \subset \GF{m_1}$ denote the set of elements whose inverses have trace $i$ for $i \in \GF{}$.
As $m_1$ is odd, $D_3$ induces a permutation of $\T_{m_1}^0$.
For $b = 1$, one can therefore compute
\begin{align*}
\sum_{u_1 \in U_1, \mulch{u_1} = 1} \addch{f_{a}(u_1)} & = \frac{1}{3} \sum_{u_1 \in U_1} \addch{f_{a}(u_1^3)} \\
& = \frac{1}{3} \sum_{u_1 \in U_1} \addch{\tr{m_1}{a \left( u_1^3 + u_1^{-3} \right)}} \\
& = \frac{1}{3} \sum_{u_1 \in U_1} \addch{\tr{m_1}{a D_3(u_1 + u_1^{-1})}} \\
& = \frac{1}{3} \left( 1 + \sum_{u_1 \neq 1} \addch{\tr{m_1}{a D_3(u_1 + u_1^{-1})}} \right) \\
& = \frac{1}{3} \left( 1 + 2 \sum_{t \in \T_{m_1}^1} \addch{\tr{m_1}{a D_3(t)}} \right) \\
& = \frac{1}{3} \left( 1 + 2 \sum_{t \in \GF{m_1}} \addch{\tr{m_1}{a D_3(t)}} - 2 \sum_{t \in \T_{m_1}^0} \addch{\tr{m_1}{a D_3(t)}} \right) \\
& = \frac{1}{3} \left( 1 + 2 \sum_{t \in \GF{m_1}} \addch{\tr{m_1}{a D_3(t)}} - 2 \sum_{t \in \T_{m_1}^0} \addch{\tr{m_1}{a t}} \right) \\
& = \frac{1}{3} \left( 1 + 2 C_{m_1}(a, a) - K_{m_1}(a) \right) \enspace .
\end{align*}
For $b \neq 1$, one deduces
\begin{align*}
\sum_{u_1 \in U_1, b \mulch{u_1} = 1} \addch{f_{a}(u_1)}
& = \frac{1}{3} \left( 1 - C_{m_1}(a, a) - K_{m_1}(a) \right) \enspace .
\end{align*}

Summarizing the above,
for $\omega = 0$:
\begin{align}
\Wa{f_{a,b}}(0)
= \left\{
\begin{array}{ll}
1 + \frac{2^{m_1}-1}{3} \left( 1 - K_{m_1}(a) - 4 C_{m_1}(a, a) \right) & \text{if $b = 1$ ;} \\
1 + \frac{2^{m_1}-1}{3} \left( 1 - K_{m_1}(a) + 2 C_{m_1}(a, a) \right) & \text{if $b \neq 1$ ;}
\end{array}
\right.
\end{align}
and for $\omega \neq 0$:
\begin{align}
\Wa{f_{a,b}}(\omega)
& = \left\{
\begin{array}{ll}
1 + 2^{m_1} \addch{f_{a,b}(w_1^{-1})} + \frac{1}{3} \left( 1 - K_{m_1}(a) - 4 C_{m_1}(a, a) \right) & \text{if $b = 1$ ;}\\
1 + 2^{m_1} \addch{f_{a,b}(w_1^{-1})} + \frac{1}{3} \left( 1 - K_{m_1}(a) + 2 C_{m_1}(a, a) \right) & \text{if $b \neq 1$.}
\end{array}
\right.
\end{align}

The values of $C_{m_1}(a, a)$ were computed by Carlitz~\cite{MR544577}
and yield a concise and easy to compute formula for the Walsh spectrum
of $f_{a,b}$.
Moreover, Charpin \etal~\cite{4595463,DBLP:journals/dm/CharpinHZ09} derived from Carlitz's results
that $C_{m_1}(a, a) = 0$ if and only if $K_{m_1}(a) \equiv 1 \pmod{3}$.
Together with the Hasse--Weil bound, it is enough to prove that
$f_{a,b}$ is (hyper-)bent if and only if $K_{m_1}(a) = 4$
as noted by Mesnager~\cite{DBLP:journals/dcc/Mesnager11}.

\section{Even case}

\subsection{General extension degree}

In this section, it is supposed that $\nu \geq 2$, \ie both $m_0$ and $m_1$ are even.
The main difference with the case $\nu = 1$ is that $3$ does now divide $2^{m_1}-1$ (in fact $2^{m_\nu}+1$) rather than $2^{m_1}+1$ and $\psi{u}$ does not depend on the value of $u_1$ (but only on that of $u_\nu$).

In particular, the computation of $\sum_{u \in U} f_{a,b}(u)$ is straightforward:
\begin{align}
\sum_{u \in U} f_{a,b}(u)
& = \prod_{i=2}^{\nu-1} \left( 2^{m_i}+1 \right) \sum_{u_1 \in U_1} \addch{f_a(u_1)} \sum_{u_\nu \in U_\nu} \addch{g_b(u_\nu)} \nonumber \\
& = \prod_{i=2}^{\nu-1} \left( 2^{m_i}+1 \right) \frac{2^{m_\nu} + 1}{3} \left( 1 - K_{m_1}(a) \right) \sum_{c \in \GF[4]{}^*} \addch{\tr{2}{b c}} \nonumber \\
& = - \prod_{i=2}^{\nu-1} \left( 2^{m_i}+1 \right) \frac{2^{m_\nu} + 1}{3} \left( 1 - K_{m_1}(a) \right) \nonumber \\
& = - \frac{2^{2^{\nu-1} m_\nu} - 1}{3 \left(2^{m_\nu} - 1\right)} \left( 1 - K_{m_1}(a) \right) \enspace . \label{eqn:sumfab}
\end{align}
Equation~\ref{eqn:walshzero} therefore becomes
\begin{align}
\Wa{f_{a,b}}(0)
& = 1 - \frac{2^{m_1} - 1}{3} \left( 1 - K_{m_1}(a) \right) \enspace .
\end{align}
As Mesnager~\cite{DBLP:journals/dcc/Mesnager11} showed,
the Hasse--Weil bound on $K_{m_1}(a)$ implies that,
if $f_{a,b}$ is bent, then $\Wa{f_{a,b}}(0) = 2^{m_1}$
and so $K_{m_1}(a) = 4$.

Equation~\ref{eqn:walshunit} becomes
\begin{align}
\Wa{f_{a,b}}(\omega)
& = 1 + \frac{2^{2^{\nu-1} m_\nu} - 1}{3 \left(2^{m_\nu} - 1\right)} \left( 1 - K_{m_1}(a) \right) \left( 1 - K_{m_1}(a) \right) + 2^{m_\nu} \sum_{u \in U, \tr[m_\nu]{m_0}{\omega u} = 0} \addch{f_{a,b}(u)} \enspace .
\end{align}
The sum over $u \in U$ can be split into smaller sums according to the extension $\GF{m_i}$ where
$\tr[m_i]{m_0}{u \omega} = 0$ with $1 \leq i \leq \nu$.

The first sum (corresponding to $i=1$) has to be dealt with separately and can be simplified as Equation~\ref{eqn:sumfab}:
\begin{align}
\sum_{u_1 = w_1^{-1}, u_2 \in U_2, \ldots, u_\nu \in U_\nu} \addch{f_a(u_1)} \addch{g_b(u_\nu)}
%& = - \left( 2^{m_2} + 1 \right) \cdots \left( 2^{m_{\nu-1}} + 1 \right) \frac{2^{m_\nu} + 1}{3} \addch{f_a(w_1^{-1})} \\
& = - \frac{2^{2^{\nu-1} m_\nu} - 1}{3\left(2^{m_\nu} - 1\right)} \addch{f_a(w_1^{-1})} \enspace . \label{eqn:sumfirst}
\end{align}

The second one (correspond to $i = 2$) is
\begin{align*}
\sum_{\substack{u_1 \neq w_1^{-1},\\ \tr[m_2]{m_0}{u_1 u_2 w_1 w_2} = 0,\\ u_3 \in U_3, \ldots, u_\nu \in U_\nu}} \addch{f_a(u_1)} \addch{g_b(u_\nu)}
& = \prod_{i=2}^{\nu-1} \left( 2^{m_i}+1 \right) \sum_{u_1 \neq w_1^{-1}} \addch{f_a(u_1)} \sum_{u_\nu \in U_\nu} \addch{g_b(u_\nu)} \\
& = - \frac{2^{2^{\nu-2} m_\nu} - 1}{3\left(2^{m_\nu} - 1\right)} \left( 1 - \addch{f_a(w_1^{-1})} - K_{m_1}(a) \right) \enspace ,
\end{align*}
together with similar terms:
\begin{align*}
\sum_{\substack{\tr[m_{i-1}]{m_0}{u_1 \cdots u_{i-1} w_1 \cdots w_{i-1}} \neq 0, \\ \tr[m_i]{m_0}{u_1 \cdots u_i w_1 \cdots w_i} = 0, \\u_{i+1} \in U_{i+1}, u_\nu \in U_\nu}} \addch{f_a(u_1)} \addch{g_b(u_\nu)}
& = - \prod_{i=2}^{i-1} 2^{m_i}  \frac{2^{2^{\nu-2} m_\nu} - 1}{3\left(2^{m_\nu} - 1\right)} \left( 1 - \addch{f_a(w_1^{-1})} - K_{m_1}(a) \right) \enspace ,
\end{align*}
until the penultimate one (corresponding to $i = \nu - 1$):
\begin{align*}
\sum_{\substack{\tr[m_{\nu-2}]{m_0}{u_1 \cdots u_{\nu-2} w_1 \cdots w_{\nu-2}} \neq 0, \\ \tr[m_\nu-1]{m_0}{u_1 \cdots u_{\nu-1} w_1 \cdots w_{\nu-1}} = 0, \\ u_\nu \in U_\nu}} \addch{f_a(u_1)} \addch{g_b(u_\nu)}
& = - \prod_{i=2}^{i-1} 2^{m_i} \frac{2^{m_\nu} + 1}{3} \left( 1 - \addch{f_a(w_1^{-1})} - K_{m_1}(a) \right) \enspace ;
\end{align*}
and they sum back up to
\begin{align}
%- \frac{2^{\left( 2^{\nu-2} - 1 \right) m_{\nu-1}}-1}{2^{m_{\nu-1}}-1} \frac{2^{m_\nu} + 1}{3} \left( 1 - \addch{f_a(w_1^{-1})} - K_{m_1}(a) \right)
& - \frac{2^{2 \left( 2^{\nu-2} - 1 \right) m_{\nu}}-1}{3 \left( 2^{m_{\nu}}-1 \right)} \left( 1 - \addch{f_a(w_1^{-1})} - K_{m_1}(a) \right) \label{eqn:summiddle}
\end{align}
as an easy induction shows.

The last sum (corresponding to $i = \nu$) can be split according to the value of
$\mulch{u_\nu}$ as in Section~\ref{sec:odd}:
\begin{align}
\sum_{\substack{\tr[m_{\nu-1}]{m_0}{u \omega} \neq 0, \\ \tr[m_\nu]{m_0}{u \omega} = 0}} \addch{f_a(u_1)} \addch{g_b(u_\nu)}
& = 2 \sum_{\substack{\tr[m_{\nu-1}]{m_0}{u \omega} \neq 0, \\ \tr[m_\nu]{m_0}{u \omega} = 0,\\ b \mulch{u_\nu} = 1}} \addch{f_a(u_1)}
 - \sum_{\substack{\tr[m_{\nu-1}]{m_0}{u \omega} \neq 0, \\ \tr[m_\nu]{m_0}{u \omega} = 0}} \addch{f_a(u_1)} \enspace , \label{eqn:sumlast}
\end{align}
where the second term is easily shown to be
\begin{align}
- \sum_{\substack{\tr[m_{\nu-1}]{m_0}{u \omega} \neq 0, \\ \tr[m_\nu]{m_0}{u \omega} = 0 }} \addch{f_a(u_1)}
%& = - 2^{m_2} \cdots 2^{m_{\nu - 1}} \left( 1 - \addch{f_a(w_1^{-1})} - K_{m_1}(a) \right) \\
& = - 2^{2 \left(2^{\nu - 2} - 1\right) m_\nu} \left( 1 - \addch{f_a(w_1^{-1})} - K_{m_1}(a) \right) \enspace . \label{eqn:sumeasy}
\end{align}

Equations~\ref{eqn:sumfirst}, \ref{eqn:summiddle}, \ref{eqn:sumlast} and \ref{eqn:sumeasy} lead to the following expression for the Walsh transform at $\omega \neq 0$:
\begin{align}
\Wa{f_{a,b}}(\omega)
& = 1 + \frac{2^{2^{\nu-1} m_\nu} - 1}{3\left(2^{m_\nu} - 1\right)} \left( 1 - K_{m_1}(a) \right) \nonumber \\
& \qquad - 2^{m_\nu} \frac{2^{2^{\nu-1} m_\nu} - 1}{3\left(2^{m_\nu} - 1\right)} \addch{f_a(w_1^{-1})} \nonumber \\
& \qquad - 2^{m_\nu} \frac{2^{2 \left( 2^{\nu-2} - 1 \right) m_{\nu}}-1}{3 \left( 2^{m_{\nu}}-1 \right)} \left( 1 - \addch{f_a(w_1^{-1})} - K_{m_1}(a) \right) \nonumber \\
& \qquad - 2^{m_\nu} 2^{2 \left(2^{\nu - 2} - 1\right) m_\nu} \left( 1 - \addch{f_a(w_1^{-1})} - K_{m_1}(a) \right) \nonumber \\
& \qquad + 2^{m_{\nu} + 1} \sum_{\substack{\tr[m_{\nu-1}]{m_0}{u \omega} \neq 0,\\ \tr[m_\nu]{m_0}{u \omega} = 0,\\ b \mulch{u_\nu} = 1}} \addch{f_a(u_1)} \nonumber \\
& = 1 + \frac{2^{2^{\nu-1} m_\nu} - 1}{3\left(2^{m_\nu} - 1\right)} \left( 1 - 2^{m_\nu} \addch{f_a(w_1^{-1})} - K_{m_1}(a) \right) \nonumber \\
& \qquad - \left( \frac{2^{\left( 2^{\nu-1} - 2^{m_\nu} \right) m_{\nu}}-1}{3 \left( 2^{m_{\nu}}-1 \right)} + 2^{\left(2^{\nu - 1} - 1\right) m_\nu} \right) \left( 1 - \addch{f_a(w_1^{-1})} - K_{m_1}(a) \right) \nonumber \\
& \qquad + 2^{m_{\nu} + 1} \sum_{\substack{\tr[m_{\nu-1}]{m_0}{u \omega} \neq 0,\\ \tr[m_\nu]{m_0}{u \omega} = 0,\\ b \mulch{u_\nu} = 1}} \addch{f_a(u_1)} \enspace . \label{eqn:walshfull}
\end{align}
Unfortunately, the remaining sum
\begin{align}
\sum_{\substack{\tr[m_{\nu-1}]{m_0}{u \omega} \neq 0,\\ \tr[m_\nu]{m_0}{u \omega} = 0,\\ b \mulch{u_\nu} = 1}} \addch{f_a(u_1)} \label{eqn:gauss}
\end{align}
is hard to make explicit.
Doing so is equivalent to evaluating a Gauss sum: an exponential sum involving
a multiplicative and an additive character.
In the next section, we manage to tackle the case $\nu = 2$ when $\omega \in \GF{m_1}^*$ (that is $w_1 = 1$) and conjecture a partial formula when $\omega \not\in \GF{m_1}^*$.

\subsection{Four times an odd number}

In this section, it is supposed that $\nu = 2$, \ie $m_0$ is four times the odd number $m_2$.

For $\tr[m_2]{m_0}{u \omega}$ to be zero with $u_1 \neq w_1^{-1}$,
$u_2$ must be $u_2 = \left( \omega_2 \tr[m_1]{m_0}{u_1 \omega_1} \right)^{-1}$
so that Equation~\ref{eqn:gauss} becomes
\begin{align*}
\sum_{u_1 \neq w_1^{-1}, \mulch{w_2 \tr[m_1]{m_0}{u_1 w_1}} = b} \addch{\tr{m_0}{a u_1^{-2}}} \\
\end{align*}
%Remark that
%\begin{align*}
%b \left( w_2^{-1} \tr[m_1]{m_0}{u_1 w_1}^{-1} \right)^{\frac{2^{m_0}-1}{3}} & = b \left( w_2^{-1} \left( u_1 w_1 + u_1^{-1}w_1^{-1} \right)^{-1} \right)^{\frac{2^{m_0}-1}{3}} \\
%& = b w_2^{-\left( 2^{m_1} + 1 \right) \left( 2^{m_2} - 1 \right) \frac{2^{m_2}+1}{3}} \left( u_1 w_1 + u_1^{-1}w_1^{-1} \right)^{- \left( 2^{m_1} + 1 \right) \left( 2^{m_2} - 1 \right) \frac{2^{m_2}+1}{3}} \\
%& = b w_2^{4 \frac{2^{m_2}+1}{3}} \left( u_1 w_1 + u_1^{-1}w_1^{-1} \right)^{- 2 \frac{2^{m_1}-1}{3}} \\
%\end{align*}

\subsubsection{The subfield case: $w_1 = 1$}
We now suppose that $w_1 = 1$, that is $\omega \in \GF{m_1}^*$ rather than $\omega \in \GF{m_0}^*$ in full generality.
Then the sum is
\begin{align*}
\sum_{u_1 \neq 1, \left( u_1^2 + u_1^{-2} \right)^{\frac{2^{m_1}-1}{3}} = b w_2^{4 \frac{2^{m_2}+1}{3}}} \addch{\tr{m_0}{a u_1^{-2}}} & = \sum_{u_1 \neq 1, \left( u_1 + u_1^{-1} \right)^{\frac{2^{m_1}-1}{3}} = b w_2^{4 \frac{2^{m_2}+1}{3}}} \addch{\tr{m_0}{a u_1^{-1}}} \\
 & = \sum_{u_1 \neq 1, \left( u_1 + u_1^{-1} \right)^{\frac{2^{m_1}-1}{3}} = b w_2^{4 \frac{2^{m_2}+1}{3}}} \addch{\tr{m_1}{a \left( u_1 + u_1^{-1} \right)}} \\
& = 2 \sum_{t \in \T_{m_1}^1, t^{\frac{2^{m_1}-1}{3}} = b w_2^{4 \frac{2^{m_2}+1}{3}}} \addch{\tr{m_1}{a t}} \\
\end{align*}

First notice that:
\begin{align*}
2 \sum_{t \in \T_{m_1}^1} \addch{\tr{m_1}{a t}} = - K_{m_1}(a) \\
\end{align*}

Define $\gamma \in \GF[4]{}^*$ by $\gamma = b w_2^{4 \frac{2^{m_2}+1}{3}}$,
let $c \in \GF[2]{m_1}^*$ be such that $\mulch{c} = \gamma$,
$\alpha = \mulch{a}$,
and let $\beta \in \GF[4]{}^*$ be a primitive third root of unity.
The sum can then be written as
\begin{align*}
2 \sum_{t \in \T_{m_1}^1, \mulch{t} = \gamma} \addch{\tr{m_1}{a t}} & = - \sum_{x \in \GF{m_1}^*, \mulch{x} = \gamma} \addch{\tr{m_1}{a x + 1/x}} + \sum_{x \in \GF{m_1}^*, \mulch{x} = \gamma} \addch{\tr{m_1}{a x}} \\
\end{align*}

Let us now proceed with the first sum.
First remark that summing over the three possible values of $\gamma$ yields
\begin{align*}
\sum_{x \in \GF{m_1}^*} \addch{\tr{m_1}{a x + 1/x}}
& = K_{m_1}(a) - 1 \\
\end{align*}

Moreover, one has
\begin{align*}
\sum_{x \in \GF{m_1}^*, \mulch{x} = \gamma} \addch{\tr{m_1}{a x + 1/x}}
& = \sum_{x \in \GF{m_1}^*, \mulch{x} = \mulch{c}} \addch{\tr{m_1}{a x + 1/x}} \\
& = \sum_{x \in \GF{m_1}^*, \mulch{x} = 1} \addch{\tr{m_1}{a c x + 1/\left( c x \right)}} \\
& = \sum_{x \in \GF{m_1}^*, \mulch{x} = 1} \addch{\tr{m_1}{a c / x + x / c}} \\
& = \sum_{x \in \GF{m_1}^*, \mulch{x} = 1} \addch{\tr{m_1}{a x / (a c) + a c / x}} \\
& = \sum_{x \in \GF{m_1}^*, \mulch{x} = \mulch{1 / a c}} \addch{\tr{m_1}{a x + 1 / x}} \\
& = \sum_{x \in \GF{m_1}^*, \mulch{x} = \alpha^2 \gamma^2} \addch{\tr{m_1}{a x + 1 / x}} \\
\end{align*}
Therefore, the first sum takes the same value at $\gamma$ and $\alpha^2 \gamma^2$.
As $\gamma$ lies in $\GF[4]{}^*$, this implies that the first sum takes the same value for $\gamma$ ranging over the two different elements of $\GF[4]{}^* \backslash \{ \alpha \}$.
If $\alpha = 1$, that is if $a$ is a cube, then this tells that the first sum takes the same value for $\gamma \in \{ \beta, \beta^2 \}$.

Denote by $r$ a square root of $a$:
\begin{align*}
\sum_{x \in \GF{m_1}^*, \mulch{x} = \mulch{1/r}} \addch{\tr{m_1}{a x + 1/x}}
& = \sum_{x \in \GF{m_1}^*, \mulch{x} = 1} \addch{\tr{m_1}{r \left( x + 1/x \right)}} \\
& = \frac{\sum_{x \in \GF{m_1}^*} \addch{\tr{m_1}{r \left( x^3 + 1/x^3 \right)}}}{3} \\
& = \frac{\sum_{x \in \GF{m_1}^*} \addch{\tr{m_1}{r D_3(x + 1/x)}}}{3} \\
& = \frac{2 \sum_{t \in \T_{m_1}^0 \setminus \set{0, 1}} \addch{\tr{m_1}{r D_3(t)}} + 1}{3} \\
& = \frac{2 \sum_{t \in \T_{m_1}^0} \addch{\tr{m_1}{r D_3(t)}} - 1}{3} \\
& = \frac{2 \left( C_{m_1}(r, r) - \sum_{t \in \T_{m_1}^1} \addch{\tr{m_1}{r D_3(t)}} \right) - 1}{3} \\
& = \frac{2 C_{m_1}(r, r) - 2 \sum_{t \in \T_{m_1}^1} \addch{\tr{m_1}{r t}} - 1}{3} \\
& = \frac{2 C_{m_1}(r, r) + K_{m_1}(r) - 1}{3} \\
& = \frac{2 C_{m_1}(a, a) + K_{m_1}(a) - 1}{3} \\
\end{align*}
The possible values of $C_{m_1}(a, a)$ were determined by Carlitz~\cite{MR544577}.
In our case where $m_1$ is even, they depend on $\mulch{a}$, and when $a$ is a cube,
whether the cube root of $a$ has half-trace zero or not.

For our interest, the most important fact is that Carlitz~\cite{MR544577} showed that
$C_{m_1}(a, a) = 0$ if and only if there exists $b \in \GF{m_1}^*$ such that $a = b^3$ (that is $a$ is a cube)
and $\tr[2]{m_1}{b} \neq 0$ (that is the cube root half-trace is non zero),
and that Charpin \etal~\cite{4595463,DBLP:journals/dm/CharpinHZ09} later deduced that 
these conditions are equivalent to $K_{m_1}(a) \equiv 1 \pmod{3}$.

For completeness, the other possibilities given by Carlitz~\cite{MR544577} follow.
When $a$ is a cube and $\tr[2]{m_1}{b} = 0$, then $C_{m_1}(a, a) = 2^{m_2 + 1} \addch{\tr{m_1}{u_0^3}}$,
where $u_0$ is any solution to $u^4 + u = b^4$,
that is $u_0 = \sum_{i=0}^{(m_2-3)/2} b^{4^{2*i+2}} + \gamma$ for any $\gamma \in \GF[4]{}$.
When $a$ is not a cube, then $C_{m_1}(a, a) = - 2^{m_2} \addch{\tr{m_1}{a u_0^3}}$,
where $u_0$ is the unique solution to $u^4 + u / a = 1$ also explicitly given by Carlitz~\cite{MR544577}:
%$u_0 = a^{(4^{m_2}-1)/3} \sum_{i = 0}^{m_2-1} a^{4^i} a^{(4^i - 1)/3}$.
$u_0 = \mulch{a} \sum_{i = 0}^{m_2-1} a^{4^i} a^{(4^i - 1)/3}$.

Therefore, in the case where $K_{m_1}(a) \equiv 1 \pmod{3}$, which includes the case we are ultimately interested in, $K_{m_1}(a) = 4$, the above equality becomes for $\gamma = 1/r = 1$:
\begin{align*}
\sum_{x \in \GF{m_1}^*, \mulch{x} = 1} \addch{\tr{m_1}{a x + 1/x}}
& = \frac{K_{m_1}(a) - 1}{3} \\
\end{align*}
Using the other equalities, it becomes evident that the first sum also takes the same value for $\gamma \neq 1$.

The second sum can also be made explicit using results of Carlitz~\cite{MR544577}.
\begin{align*}
\sum_{x \in \GF{m_1}^*, \mulch{x} = \gamma} \addch{\tr{m_1}{a x}}
& = \sum_{x \in \GF{m_1}^*, \mulch{x} = \mulch{c}} \addch{\tr{m_1}{a x}} \\
& = \sum_{x \in \GF{m_1}^*, \mulch{x} = 1} \addch{\tr{m_1}{a c x}} \\
& = \frac{1}{3} \sum_{x \in \GF{m_1}^*} \addch{\tr{m_1}{a c x^3}} \\
& = \frac{1}{3} C_{m_1}(a c, 0) \\
\end{align*}
As $m_1$ is even, this becomes
\begin{align*}
\sum_{x \in \GF{m_1}^*, \mulch{x} = \gamma} \addch{\tr{m_1}{a x}}
& = \left\{
\begin{array}{ll}
\frac{-1 + (-1)^{m_2-1} 2^{m_2+1}}{3} & \text{if } \mulch{a c} = 1 \\
\frac{-1 + (-1)^{m_2} 2^{m_2}}{3} & \text{if } \mulch{a c} \neq 1 \\
\end{array}
\right. \\
\end{align*}
In particular, its value only depends on $\mulch{a c} = \alpha \gamma$.

To summarize, supposing that $K_{m_1}(a) \equiv 1 \pmod{3}$, one has
\begin{align*}
\sum_{u_1 \neq 1, \left( u_1 + u_1^{-1} \right)^{2 \frac{2^{m_1}-1}{3}} = 1} \addch{\tr{m_0}{a u_1^{-2}}} = \frac{2^{m_2+1} - K_{m_1}(a)}{3} \\
\end{align*}
and so that
\begin{align*}
& \sum_{u_1 \neq 1, \left( u_1^2 + u_1^{-2} \right)^{\frac{2^{m_1}-1}{3}} = b w_2^{4 \frac{2^{m_2}+1}{3}}} \addch{\tr{m_0}{a u_1^{-2}}} \\
& \qquad = \frac{2^{m_2+1} - K_{m_1}(a)}{3} - \left( 1 - \addch{\tr{2}{b w_2^{4 \frac{2^{m_2}+1}{3}}}} \right) 2^{m_2-1} \\
\end{align*}

\subsection{A conjectural formula: $w_1 \neq 1$}
Going back to the case $w_1 \neq 1$, \ie $\omega \in \GF{m_0}^*$, and supposing again that $K_{m_1}(a) \equiv 1 \pmod{3}$, the following seems to be true experimentally:
\begin{align*}
& \sum_{u_1 \neq w_1^{-1}, \left( (u_1 w_1)^2 + (u_1 w_1)^{-2} \right)^{\frac{2^{m_1}-1}{3}} = b w_2^{4 \frac{2^{m_2}+1}{3}}} \addch{\tr{m_0}{a u_1^{-2}}} \\
& \qquad = \frac{2^{m_2+1} - K_{m_1}(a)}{3} - \left( 1 - \addch{\tr{??}{??}} \right) 2^{m_2-1} - \left( 1 - \addch{\tr{m_0}{a w_1^2}} \right) \frac{2^{m_2-1} - 1}{3} \\
\end{align*}

Assuming $K_{m_1}(a) \equiv 1 \pmod{3}$ and the above formula for the sum is correct and plugging it into Parseval equality yields the following equality:
\begin{align*}
\sum_{x \in \GF{m_0}^*} \addch{\tr{??}{??}} & = \frac{2^{m_1} - 1}{3} \left( K_{m_1}(a) - 1 \right) \\
& = \Wa{f_{a,b}}(0) - 1 \\
\end{align*}
This is supported by experimental evidence that there are exactly $2^{m_1 - 1} + 5 \frac{K_{m_1}(a) - 4}{6} + 3$ (respectively $2^{m_1 - 1} - \frac{K_{m_1}(a) - 4}{6}$) values of $w_1 \in U_1$ such that the mysterious trace is zero when $b \mulch{w_2} = 1$ (respectively $b \mulch{w_2} \neq 1$).


%Indeed
%\begin{align*}
%& \sum_{u_1 \neq w_1^{-1}, \left( (u_1 w_1)^2 + (u_1 w_1)^{-2} \right)^{\frac{2^{m_1}-1}{3}} = b w_2^{4 \frac{2^{m_2}+1}{3}}} \addch{\tr{m_0}{a u_1^{-2}}} \\
%& \qquad = \sum_{u_1 \neq 1, \left( u_1^2 + u_1^{-2} \right)^{\frac{2^{m_1}-1}{3}} = b w_2^{4 \frac{2^{m_2}+1}{3}}} \addch{\tr{m_0}{a u_1^{-2} w_1^2}} \\
%& \ldots
%\end{align*}

\subsubsection{An explicit expression for the Walsh transform}

In the end we get the following (conjectured) formula when $K_{m_1}(a) \equiv 1 \pmod{3}$ and $\omega \neq 0$:
\begin{align*}
\Wa{f_{a,b}}(\omega)
& = 1 + \frac{2^{m_2} + 1}{3} \left(1 - 2^{m_2} \addch{f_a(w_1^{-1})} - K_{m_1}(a) \right) \\
& \qquad - 2^{m_2} \left(1 - \addch{f_a(w_1^{-1})} - K_{m_1}(a) \right) \\
& \qquad + 2^{m_2 + 1} \left( \frac{2^{m_2 + 1} - K_{m_1}(a)}{3}  - \left( 1 - \addch{\tr{??}{??}} \right) 2^{m_2-1} - \left( 1 - \addch{f_a(w_1^{-1})} \right) \frac{2^{m_2-1} - 1}{3} \right) \\
& = 2^{m_1} \addch{\tr{??}{??}} + \frac{4 - K_{m_1}(a)}{3}
\end{align*}

This implies in particular that $f_{a,b}$ is bent if and only if $K_{m_1}(a) = 4$.
(The above formula is valid only for $K_{m_1}(a) \equiv 1 \pmod{3}$ but recall that the value of the Walsh transform at $0$ enforces $K_{m_1}(a) = 4$ in all cases.)

\section{Conclusion}

\bibliographystyle{plain}
\bibliography{even}

\end{document}
